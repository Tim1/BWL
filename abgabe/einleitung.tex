\clearpage
\pagenumbering{arabic}
\section{Einleitung}
Elektronisches Bezahlen, \emph{E-Payment}, kann Kunden das Leben einfacher machen und Dienstleistern Bearbeitungskosten erspraren sowie die Kundenbindung\footnote{\url{https://en.wikipedia.org/wiki/Customer\_retention}} verbessern. Das Spektrum an E-Payment-Anbietern und \mbox{-Lö}sungen ist groß wie divers und deren Funktionsweise nicht immer direkt ersichtlich. Daher sollen im Folgenden das E-Payment selbst sowie mehrere verschiedene Dienstleister dieser Branche beleuchtet werden.\\
Hierzu ist es vorab nötig, die wesentlichen für Bescheibung und Analyse erforderlichen Begrifflichkeiten zu definieren und abzugrenzen. Anschließend werden wir wichtige Technologien vorgestellen, mit welchen E-Payment realisiert wird, und auf die E-Payment-Wertschöpfungskette eingehen. Im Kapitel \emph{Provider} werden sechs verschiedene E-Payment-Anbieter bzw. \mbox{-Lö}sungen beschrieben und aufgezeigt, wie diese Gewinne erzielen. Abschließend schil-dern wir die Position genannter Provider im Markt, deren Konkurrenzstellung sowie eine Aussicht der zukünftigen Entwicklung des Marktes und ziehen ein Fazit.
