\clearpage

\section{Definitionen}

\subsection{E-Paymnet}

\begin{itemize}

	\item bezeichnet eine neue Erscheinungsform des Geldes.
	\item 2 Kategorien von E-Geld:
    
    \begin{itemize}
    	\item kartengesch�tzes E-Geld (Kartengeld), z.B Kontokarten mit Chip oder Magnetstreifen
        \item softwarebasiertes E-Geld (Netzgeld), z.B Online-Konten, die E-Geld f�r den Kunden transferiert.

    \end{itemize}

	\item Elektronische Zahlungssysteme
 	\begin{itemize}
    	\item kategorisiert je nach Betrachtungsweise:
        \begin{itemize}
    		\item Zeitpunkt (Prepaid, Pay Now, Pay Later)
            \item H�he des Betrags (Micro, Macto-Payment)
		\end{itemize}
	\end{itemize}
    
    
    
    \item Anforderung an E-Payment (F�lschungssicherheit,einfach, schnell, sicher..)
\end{itemize}




\subsection{Gesch�ftsmodell}

\begin{itemize}
	\item beschreibt die logische Funktionsweise eines Unternehmens und insbesondere die spezifische Art und Weise, mit er es Gewinne erwirtschaftet.

	\item hilft Schl�sselfaktoren des Unternehmenserfolges oder Misserfolges zu verstehen und zu analysieren.

    \item Ein Gesch�ftsmodell besteht aus 3 Hauptkomponenten
    \begin{itemize}
    	\item Nutzenversprechen: welchen Nutzen und Wert stiftet das Unternehmen f�r Kunden und strategische Partner?
        \item Architektur der Wertsch�pfung: wie wird die Leistung in welcher Konfiguration erstellt? 
        \item Ertragsmodell: Wodurch wird das Geld verdient?
    \end{itemize}
    
    
    \item Kernkompetenzen
    \begin{itemize}
    	\item Eigenschaften einer Kompetenz:
        
        \begin{itemize}
            \item Differenzierung
            \item Diversifikation
            \item Kundennutzen
            \item Imitatioinsschutz
        \end{itemize}
    \end{itemize}


\end{itemize}



\subsection{Strategie}


\begin{itemize}
    \item meist langfristig geplante Verhaltensweisen der Unternehmen zur Erreichung ihrer Ziele
	\item die verschiedenen Strategien:
    \item ...
    \begin{itemize}
    	\item Klassische Strategie
        \item Unternehmerische Strategie
        \item Ideologische Strategie
        \item Prozess Strategie
	\end{itemize}
    
    \item ...
    
\end{itemize}


\subsection{Kennzahlen}

\begin{itemize}
    \item Betriebswirtschaftliche Kennzahlen werden innerhalb der Betriebswirtschaft zur Beurteilung von Unternehmen eingesetzt.

    \item Dient als Basis f�r:
        \begin{itemize}
      	  	\item Entscheidungen (Problemerkennung, Ermittlung von betrieblichen Stark- und Schwachstellen, Informationsgewinnung) 
			\item Kontolle (Soll-Ist-Vergleich)
            \item Dokumentation
            \item Kooridnation wichtiger Sachverh�ltnisse
    	\end{itemize}

    \item Arten von Kennzahlen
    \begin{itemize}
        \item Absolute Kennzahlen
        \item Relative Kennzahlen
    \end{itemize}

    \item Funktionen von Kennzahlen
    \item Gliederung von Kennzhalen nach dem zugrunde liegeneden Sachverhaltes:
    \begin{itemize}
        \item Erfolgskennzahlen
        \item Liquidit�tskennzahlen
        \item Rentabilit�tskennzahlen
      	\item Kennzahlen zur Kapitalstruktur
    \end{itemize}
    \item ...
\end{itemize}






\subsection{Indikatoren}

\begin{itemize}
    \item volkswirtschaftlicher Indikator (Konjunktur oder makro�konomische Kennzahl) ist eine Messgr��e, die Aussagen �ber die konjunkturelle Entwicklung oder die wirtschaftliche Situation im Allgemeinen von Volkswirtschaften erlaubt.

    \item Betriebswirtschaftliche Indikatoren sind die betriebswirtschaftliche Kennzahlen, die vorhin im Kapitel Kennzahlen erl�utert wurden.

    \item Konjunkturindikatoren werden h�ufig auch bei der Bewertung von Aktien eingesetzt, da aus der gesamt-volkswirtschaftlichen Entwicklung R�ckschl�sse auf die Entwicklung einzelner Industriesektoren gezogen werden k�nnen, die wiederum die unternehmerischen Erfolgsaussichten von einzelnen Unternehmen beeinflussen.
 
    \item Drei Kategorien der Konjunkturindikatoren:
    \begin{itemize}
        \item Mengen- und Preis- bzw. Kostenindikator
        \item Fr�h-, Pr�senz und Sp�tindikatoren: nach zeitlichen Verlauf eines Sachverhaltes
		\item Wachstumsrate (Inflationsrate)
    \end{itemize}
\end{itemize}
