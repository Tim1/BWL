\section{Markt- und Konkurrenz-Analyse}

\subsection{Marktsituation}
\begin{itemize}
\item Barzahlen ist ein neuer Anbieter mit bisher 1425 Verkaufsstellen.
\item Paysafecard ist mit 450.000 Verkaufsstellen in 33 Ländern europäischer Marktführer bei online Prepaid Zahlungsmitteln.
\item Paypal ist mit 230 Millionen Mitgliedskonten in 190 Nationen Marktführer des Payment Processing existiert seit 1998.
\item Google Wallet funktioniert ähnlich wie Paypal, unterstützt aber auch den NFC-Standard. Momentan gibts es ca. 15 unterstützte NFC-Handys und Tablets und 300.000 Mastercard-paypass Stores. Google Wallet ist seit 2011 auf dem Markt aktiv.
\item Amazon Payments wurde 2007 gegründet und ist komplett in Amazon integriert.
\end{itemize}

\subsection{Analyse}
Im Bereich des E-Payments gibt es bei den von uns untersuchten Providern Dienstleistungsunternehmen mit verschiedenen Geschäftsmodellen.\\

Paysafecard und Barzahlen haben sich als Ziel gesetzt, einen Dienst anzubieten, welcher auf das Preisgeben von persönlichen Daten und das Nutzen von Kreditkarten und Bankkontodaten verzichtet. Dies wird von diesen beiden Providern durch unterschiedliche Wege erzielt. Dabei ist zu beachten, dass Paysafecard momentaner Marktführer bei online Prepaid Zahlungsmitteln ist und Barzahlen erst seit kurzem existiert. Barzahlen wird es daher anfangs schwer haben am Markt Fuß zu fassen, da sie um einiges weniger Verkaufsstellen haben als Paysafecard. Dazu ist noch zu erwähnen, dass Barzahlen nur in Deutschland aktiv ist. Aus diesem Grund ist Paysafecard wohl momentan um einiges erfolgreicher.\\

Die beiden oben beschriebenen Provider funktionieren nur, da man vorher bzw. nachher an einer Filiale etwas mit Bargeld bezahlen muss. Ein anderes Ziel haben sich Paypal, Google Wallet, Amazon Payments etc. gesetzt. Diese wollen dem Kunden einen einfachen Weg anbieten, um im online einzukaufen. Dies wird allerdings über Kreditkarten bzw. Bankkontodaten ermöglicht. Zusätzlich ist die Registrierung eines eigenen Accounts erforderlich. Hier gibt es mehr bekannte Provider, die miteinander konkurrieren. Zum einen ist es Paypal, der momentane europäische Marktführer in diesem Bereich. Paypal hat den großen Vorteil, dass es zu eBay gehört und dort kein anderer Dienst akzeptiert wird. Dadurch hat Paypal automatisch schon einen großen Teilmarkt für sich im Besitz genommen. Des weiteren existiert Paypal schon lange und konnte Erfahrungen sammeln und diese in ihre Strategie berücksichtigen.\\

Ein anderer Konkurrent ist Google Wallet. Dieser bietet den Vorteil, dass es komplett in gmail integriert ist und man sozusagen per Mail Geld verschicken kann. Anders als andere Dienstleistungsanbieter, verlangt Google keine Gebühren. Der Gewinn wird durch Werbung erzielt. Zusätzlich bietet Google Wallet noch die Möglichkeit an, mobil per NFC zu bezahlen. In diesem Bereich existieren aber bisher noch nicht so viele verschiedene Handys bzw. Tabletts.\\

Noch ein immer größer werdender Konkurrent, welcher zu einer großen Firma gehört, ist Amazon Payments. Dieser Provider hat den Vorteil, dass durch den immer beliebter werdenden Händler immer mehr Kunden dazu kommen, da jeder, der bei Amazon registriert ist, automatisch Amazon Payments nutzen kann.\\

Momentan sieht es so aus, dass Paypal noch führend ist.

\subsection{Potenziale}
Momentan ist der Markt bezüglich mobilem Zahlen sehr interessant, da Google Wallet dies schon anbietet und auch unter anderem schon Paypal angekündigt hat, dass sie auch das Zahlen per NFC an der Kasse ermöglichen wollen. Dies könnte also eine interessante neue Möglichkeit bieten. Das einzige Problem momentan ist wohl, dass es noch zu wenig Handys oder Tabletts gibt, die dies derzeit unterstützen. Doch dies wird in der Zukunft sicherlich kein großes Hindernis darstellen.


\subsection{Aussicht}
In Zukunft könnte der Markt im Bereich des Bezahlens ohne Kreditkarte bzw. ohne Bankkontodaten interessanter werden, daher wäre es gut möglich, dass es hier noch mehr Konkurrenz geben wird. Bei den jetzigen Providern in diesem Bereich wird wahrscheinlich Paysafecard weiterhin eine führende Position haben, da sie schon sehr stark verbreitet sind, dadurch dass sie schon lange existieren. Ob Barzahlen zu einem ernsthaften Konkurrent werden kann, ist nur davon abhängig, ob sie sich gut verbreiten können. Zusätzlich brauchen sie auch genug Händler, die diesen Zahlungsart anbieten.\\

Im anderen Bereich könnte es sehr interessant werden, da derzeit sehr kritisiert wird, dass nur Paypal bei eBay genutzt werden kann. Außerdem hat Paypal auch große und bekannte Konkurrenz bekommen, die auch schon eine große Anzahl an potenziellen Kunden hat. Hier könnte sich in Zukunft viel tun.\\

Zusätzlich wird hier auch das mobile Bezahlen Auswirkungen haben.