\section{Markt- und Konkurrenz-Analyse}

\subsection{Marktsituation}
\begin{itemize}
\item Barzahlen ist ein neuer Anbieter mit bisher 1425 Verkaufsstellen.
\item Paysafecard ist mit 450.000 Verkaufsstellen in 33 Ländern europäischer Marktführer bei online Prepaid Zahlungsmitteln.
\item Paypal ist mit 230 Millionen Mitgliedskonten in 190 Nationen Marktführer des Payment Processing. Es existiert seit 1998.
\item Google Wallet funktioniert ähnlich wie Paypal, unterstützt aber auch den NFC-Standard. Es existiert seit 2011. Momentan gibts es ca. 15 unterstützte NFC-Handys und Tablets und 300.000 Mastercard-paypass Stores.
\item Amazon Payments wurde 2007 gegründet und ist komplett in Amazon integriert.
\end{itemize}

\subsection{Analyse}
Im Bereich des e-Payments gibt, es bei den von uns untersuchten Providern Dienstleistungsunternehmen mit verschiedenen Geschäftsmodellen.\\

Paysafecard und Barzahlen haben sich als Ziel gesetzt, einen Dienst anzubieten, welcher auf das Preisgeben von persönlichen Daten und das Nutzen von Kreditkarten und Bankkontodaten verzichtet. Dies wird von diesen beiden Providern durch unterschiedliche Wege erzielt. Dabei ist zu beachten, dass Paysafecard momentaner Marktführer bei online Prepaid Zahlungsmitteln ist und Barzahlen erst seit kurzem existiert. Daher wird es Barzahlen wohl erst sehr schwer haben, da sie um einiges weniger Verkaufsstellen haben wie Paysafecard. Dazu sei auch noch zu beachten, dass Barzahlen nur in Deutschland aktiv ist. Aus diesem Grund ist Paysafecard wohl momentan um einiges erfolgreicher.\\

Die beiden oben beschriebenen Provider funktionieren nur, da man vorher bzw. nachher an einer Filiale etwas mit Bargeld bezahlen muss. Ein anderes Ziel haben sich Paypal, Google Wallet, Amazon Payments etc. gesetzt. Diese wollen dem Kunden einen einfachen Weg bieten, um im Internet Dinge einzukaufen. Dies wird allerdings über Kreditkarten bzw. Bankkontodaten ermöglicht. Zusätzlich ist die Registrierung eines eigenen Accounts erforderlich. Hier gibt es mehr bekannte Provider, die miteinander konkurrieren.\\

Zum einen wäre hier Paypal, der momentane europäische Marktführer in diesem Bereich. Paypal hat den großen Vorteil, dass es zu eBay gehört und dort kein anderer Dienst akzeptiert wird. Dadurch hat Paypal automatisch schon einen großen Vorteil. Desweiteren gibt es Paypal auch schon sehr lange und es konnte sich dadurch schon sehr gut verbreiten.\\

Ein anderer Konkurrent ist Google Wallet. Diese bieten den Vorteil, dass es komplett in gmail integriert ist und man sozusagen per Mail Geld verschicken kann. Anders wie die anderen Dienstleistungsanbieter verlangt Google keine Gebühren. Der Gewinn wird durch Werbung erzielt. Zusätzlich bietet Google Wallet noch die Möglichkeit an, mobil per NFC zu bezahlen. In diesem Bereich existieren aber bisher noch nicht so viele verschiedene Handys bzw. Tablets.\\

Noch ein immer größer werdender Konkurrent, welcher zu einer großen Firma gehört, ist Amazon Payments. Dieser Provider hat den Vorteil, dass durch den immer beliebter werdenden Händler immer mehr Kunden dazu kommen, da jeder, der bei Amazon registriert ist, automatisch Amazon Payments nutzen kann.\\

Momentan sieht es so aus, dass Paypal noch führend ist.

\subsection{Potenziale}
Momentan ist der Markt bezüglich mobilem Zahlen sehr interessant, da Google Wallet dies schon anbietet und auch unter anderem schon Paypal angekündigt hat, dass sie auch das Zahlen per NFC an der Kasse ermöglichen wollen. Dies könnte also einen interessante neue Möglichkeiten bieten. Das einzige Problem momentan ist wohl, dass es noch zu wenig Handys oder Tablets gibt, die dies derzeit unterstützen. Hier wird es aber wahrscheinlich auch Änderungen geben.


\subsection{Aussicht}
In Zukunft könnte der Markt im Bereich des Bezahlens ohne Kreditkarte bzw. ohne Bankkontodaten interessanter werden, daher wäre es gut möglich, dass es hier noch mehr Konkurrenz geben wird. Bei den jetzigen Providern in diesem Bereich wird wahrscheinlich Paysafecard weiterhin eine führende Position haben, da sie schon sehr stark verbreitet sind, dadurch dass sie schon lange existieren. Ob Barzahlen zu einem ernsthaften Konkurrent werden kann, ist nur davon abhängig, ob sie sich gut verbreiten können. Zusätzlich brauchen sie auch genug Händler, die diesen Bezahlungsweg anbieten.\\

Im anderen Bereich könnte es sehr interessant werden, da derzeit sehr kritisiert wird, dass nur Paypal bei eBay genutzt werden kann. Außerdem hat Paypal auch große und bekannte Konkurrenz bekommen, die auch schon eine große Anzahl an potenziellen Kunden hat. Hier könnte sich in Zukunft viel tun.\\

Zusätzlich wird hier auch das mobile Bezahlen Auswirkungen haben.