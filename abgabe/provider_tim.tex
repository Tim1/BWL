\subsection{Google Wallet}
\subsubsection{Das Unternehmen}
Google Wallet ist ein Online-Bezahlsystem des Unternehmens Google Inc. Vorgestellt im Mai 2011 startete der Dienst offiziell im September 2011.
Damit stellt Google Wallet außerdem auch den Nachfolger von Google Checkout dar, welcher nach Übergangszeit nach und nach ersetzt wird.\footnote{\url{http://www.google.com/wallet/index.html}}

\subsubsection{ Geschäftsmodell}
\textbf{Funktionsweise}\\
Wie der Name aussagt stellt Google Wallet eine "Brieftasche" dar in mehrerer verschiende Kredit- oder Debitkarten hinterlegt werden können.
Als eins der Hauptmerkmal des Dienstes steht das mobile Bezahlen durch ein Android-Smartphone mit NFC-Chip da. Dies ist jedoch vererst nur bei bestimmten Handymodellen möglich. \footnote{\url{http://support.google.com/wallet/bin/answer.py?hl=en\&answer=1347934}}
Geworben wird mit der Sicherheit sowohl für den Kunden als auch für den Händler. Durch ständige Betrugsüberwachung ("24/7 fraud monitoring") um jederzeit ungewöhnliche Aktivitäten festzustellen. Weitere Sicherheit für den Kunden soll es dadurch geben, dass alle Daten verschlüsselt gespeichert und übertragen werden. Um mobil zu Bezahlen muss sich das Handy Zentimeter vom PayPass-NFC-Reader befinden und zusätzlich über einen PIN bestätigt werden.\\
Dennoch gibt es auch kritische Stimmen. So wurde z.B. die Möglichkeit das sogenannte "eavesdropping" über die Verwendung von Google Analytics kritisiert, womit es möglich sein kann PINs auszulesen.  \footnote{\url{http://www.itworld.com/mobile-wireless/248596/even-after-rewrites-google-wallet-retains-gaping-security-holes-mainly-due-an}}\\


\textbf{Gewinne erwirtschafen}\\
Derzeit belastet Google Wallet den Benutzer mit keinerlei Gebühren oder Transaktionsabgaben. Angeleht an das Geschäftsmodell vieler Google-Prokukte wird mithilfe von Werbung Gewinn erwirtschaftet.
Ausnahme ist hierbei der spezielle Email-Dienst GMail, über den Wallet-Benutzer Geld per Email versenden können. Hier wird für Addressaten ohne Wallet Account eine Abgabe von 2,9\% berechnet.\footnote{\url{http://www.google.com/wallet/send-money/}}\\

	
\subsubsection{Strategie}
Google Inc. versucht mit Google Wallet den Zusammenschluss verschiedener Bezahlmöglichkeiten, daher wurde auch Google Checkout ersetzt.
Mit Hilfe der großen Marktmacht des Unternehmen, unter Anderem auch im mobilen Markt mit dem Betriebsystem Android, versucht es den Dienst populärer zu machen. Dabei spielt wie schon erwähnt das mobile Betriebsystem Andoid - mit einem derzeitigen Marktanteil unter Smartphones von ca. 75\% - eine große Rolle. Hier ist es im Interesse von Google die Nahfunk-Technologie NFC weiter zu verbreiten. So wurde extra für solche Anwendungen mit dem "Android Secure Element" \footnote{\url{https://code.google.com/p/seek-for-android/}} ein neues Modul zum sicheren Speichern von sensiblen Informationen entwickelt.


\subsubsection{Kernkompetenzen}
\textbf{Differenzierung}\\
Im mobilen Bereich mit der Verwendung der Nahfunk-Technologie NFC ist Google Wallet einer der frühen und wenigen Anbieter.\\
Auch das Senden von Geld per Email über einen GMail-Account stellt ein Novum dar.\\

\textbf{Diversifikation}\\
Typisch für Google ist eine Fächerung des Angebots. So kann Wallet "ganz normal" online als Webanwendung genutzt werden, mobil ist es durch eine Andoid-App im Einsatz. Mit dem Versenden von Geld per Email bietet Google nochmals eine neue Möglichkeit des ePayments an.\\

\textbf{Imitationsschutz}\\
Im Bereich Web gibt es viel Konkurrenz daher wird besonders der mobile Bereich verstärkt fokussiert. Besonders eine Anbindung an den Android Playstore ist natürlich hilfreich.\\

\textbf{Kundennutzen}\\
Beworben wird vor allem der Nutzen als "Brieftasche", durch das Hinterlegen von verschiedenen Kreditkarten. So ist auch ein ferngesteuertes Deaktivieren des Accounts möglich, falls z.B. das Smartphone verloren oder gestohlen wurde. Auch eine Auflistung aller Bezahlungen ist über eine Webanwendung möglich.\\
Auch die immer größer werdende Verbreitung des Dienstes, sowie die intergrierung in den Android Playstore führt zu immer mehr Angeboten.\\

\subsubsection{ Kennzahlen}
Mobiles Bezahlen ist derzeit nur in den Vereinigten Staaten von Amerika möglich. Hier gibt es inzwischen über 300.000 "Mastercard-paypass Stores", welche Google Wallet als Zahlungsart aktzeptieren.\footnote{\url{http://www.google.com/wallet/buy-in-store/}}
Derzeit ist die App auf 15 Smartphone und Tabletmodellen möglich. Dazu gehören neben der Google-eigenen "Nexus-Reihe" z.B. auch das Samsung Galasy S3 oder S4. 


\subsection{Amazon Payments}
Amazon Payments ist eine 100\%-ige Tocherfirma von Amazon.com. 2007 gegründet, bietet sie verschiedene Online-Bezahlsysteme an und nutzt dabei den großen Kundenstamm des Mutterunternehmen. Da alle Zahlungsmöglichkeiten des Online-Händlers auch über Amazon Payments angeboten werden, erfährt der Kunde die gleiche "checkout experience" wie auf Amazon.com.\footnote{\url{https://payments.amazon.com/sdui/sdui/home}}

\subsubsection{ Geschäftsmodell}
Amazon Payments bietet hauptsächlich zwei Bezahldinste an "Checkout by Amazon" und "Amazon Simple Pay".\\

\textbf{Checkout by Amazon (CBA)}\\
CBA bietet für Webshopanbieter die Möglichkeit Amazon für den Bezahlvorgang zu verwenden. Dabei muss der Verkäufer auf seiner Seite ein Button einbauen, welchen der Kunde dann im Warenkorb zum Bezahlen auswählen kann. Dabei bleibt der Kunde permanent auf der ursprünglichen Seite und wird nur durch ein Pop-Up Fenster gebeten seine Amazon Accountdaten anzugeben. Hat er sich dann angemeldet, so findet er die alle zuvor auf Amazon.com angegeben Bezahlmethoden und kann auch zwischen den gespeicherten Lieferaddresse auswählen.\footnote{\url{https://payments.amazon.com/sdui/sdui/business/cba}}\\

\textbf{Amazon Simple Pay (ASP)}\\
ASP ähnelt CBA in vielen Punkten, besonders hinsichtlich der "checkout experience". Jedoch wird hierbei lediglich der simple Bezahlvorgang über Amazon abgewickelt. Lieferaddressen, Werbung oder zusätzliche Kosten durch Steuern etc. werden nicht berücksichtigt.
\footnote{\url{https://payments.amazon.com/sdui/sdui/business/asp}}\\
	
\textbf{Gewinne erwirtschafen}\\
Beide Dienste erwirtschafen Gewinn mithilfe Transaktionsabgaben. Diese betragen Standartmäßig 2.9\% + \$0.30.\\
Jedoch bietet Amazon Rabatte hohen Umsatz. Beträgt der monatliche Umsatz etwa über 100.000\$ so sind nur noch 1.9\%  Transaktionsabgaben fällig.
Ebenfalls gibt es für Micropayment (5\% + \$0.05) oder Non-profit Oranisationen (2.2\% + \$0.30) verschiedene Werte.\footnote{\url{https://payments.amazon.com/sdui/sdui/helpTab/Checkout-by-Amazon/Creating-Managing-Your-Account/Amazon-Payments-Fees}}

\subsubsection{Strategie}
Amazon profitiert besonders durch die große Kundenbasis seiner Shopping-Website. Hierbei ist eine der großen Vorteil das Vorhandensein von Bezahlinformationen der Kunden. Daher ermöglicht es dem Kunden eine schnelle und einfache Arte zu bezahlen, vorausgesetzt er ist schon Kunde von Amazon.com. Weiterhin kann Amazon durch die bekannte Benutzerführung Vertrauen beim Kunden gewinnen, welches sich schließlich auf den Webshop-Anbieter auszahlt. Daneben wird versucht durch die leichte Integrierbarkeit in eigene Webseiten zusätzliche Anreize zu schaffen. 

\subsubsection{Kernkompetenzen}
\textbf{Differenzierung}\\
Hier hat Amazon Payments den Vorteil der vorhanden Bezahlinformationen durch dem Mutterkonzern. Auch Lieferadressen sind schon eingetragen. Auch durch den Vertrauten Ablauf des Bezahlvorgangs kann es punkten.\\

\textbf{Diversifikation}\\
Amazon Payments bietet seine zwei großen Dienste für jeden Verkäufer für geringe Transaktionskosten und mit simler Einbindung an. Dadurch sind die Bezahlmethoden schon weit verbreitet.\\

\textbf{Kundennutzen}\\
Für Kunden von Amazon ist der Vorgang sehr komfortabel, da keine neuen Bezahldaten eingegeben werden müssen.\\