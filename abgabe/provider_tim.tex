\subsection{ Google Wallet [15]}
\subsubsection{ Das Unternehmen}
	\begin{itemize}
	\item Google Inc.
	\item präsentiert May 26, 2011 -- gestartet (app) September 19, 2011
	\end{itemize}

\subsubsection{ Geschäftsmodell}
\textbf{Funktionsweise}\\
	\begin{itemize}
	\item Kreditkarten/Debitkarten hinterlegt in Account
	\item Mobil (nur bei bestimmten Modellen möglich [17]) und im Web bezahlen möglich
	\item 24/7 fraud monitoring. Instant transaction notifications. 

	\item Sicherheit: User-Infos verschlüsselt, zusätzlicher PIN notwendig vor dem Bezahlen, muss PayPass-NFC-Reader berühren, über Website handy disable möglich
	\item Sicherheits-Risiko: möglichkeit des eavesdropping bei Google Analytics
	\end{itemize}


\textbf{Gewinne erwirtschafen}\\
	\begin{itemize}
	\item keine Abgaben sondern durch Werbung
	\item über Gmail Geld versenden: für wallet-account kostenlos, sonst 2,9\% abgabe
	\end{itemize}


\subsubsection{ Strategie}
	\begin{itemize}
	\item Zusammenlegung/Konzentation auf ein Dienst (Einstellung von Google-Checkout)
	\item NFC ausbreiten, Android als Plattform (Android Secure element[13])
	\item Android Market
	\item Macht der Größe des Unternehmens nutzen 
	\end{itemize}

\subsubsection{ Kernkompetenzen}
\textbf{Differenzierung}\\
	\begin{itemize}
	\item NFC
	\item Gmail geld per email
	\end{itemize}
\textbf{Diversifikation}\\
	\begin{itemize}
	\item breit gefächertes Angebot (Gmail, Checkout, NFC)
	\end{itemize}
\textbf{Imitationsschutz}\\
	\begin{itemize}
	\item im Web viel Konkurrenz daher:
	\item neü Bereiche wie NFC und Gmail-Geld
	\end{itemize}
\textbf{Kundennutzen}\\
	\begin{itemize}
	\item Wallet-Funktion: 
	\subitem  mehrere Karten hinterlegen
	\subitem  deaktivieren falls gestolen o.ä.
	\subitem  nachprüfen der Bezahlungen im Web
	\subitem  großer Anbieter --> viele Angebote (steigend)
	\end{itemize}

\subsubsection{ Kennzahlen}
	\begin{itemize}
	\item 300,000+ Mastercard-paypass Stores
	\item ca. 15 NFC-Handys und Tablets (Nexus, Samsung Galaxy S3-S4)[17]
	\end{itemize}




\subsection{ Amazon Payments [16]}
\subsubsection{ Das Unternehmen}
	\begin{itemize}
	\item Tochterfirma von Amazon.com 100\%
	\item 2007 gegründet
	\item alle Zahlinformationen der Mutterfirma, gleiche "checkout experience" wie auf Amazon.com
	\end{itemize}


\subsubsection{ Geschäftsmodell}
\textbf{Funktionsweise}\\
	\begin{itemize}
	\item "checkout by Amazon" (CBA)
	\subitem  direkt auf eigener Website einbinden nur Button (website nie verlassen)
	\subitem  Bezahlmethoden und Adressen von Amazon.com nutzen, keine Kreditkarte o.ä. wieder angeben
	\subitem  Anmeldung bei amazon über pop-up
	\item "Amazon Simple Pay"
	\subitem  ähnlich wie CBA
	\subitem  nur Bezahlprozess, keine Addressen oder sonstiges
	\end{itemize}
	
\textbf{Gewinne erwirtschafen}\\
	\begin{itemize}
	\item Transaktionsabgaben: 2.9\% + \$0.30
	\item Rabatt für große monatliche Raten. Bei 100.000\$ -> nur noch 1.9\%
	\item besonderes für Micropayment: 5\% + \$0.05
	\item besonderes für Non-profit: 2.2\% + \$0.30
	\end{itemize}


\subsubsection{ Strategie}
	\begin{itemize}
	\item große Kundenbasis von amazon.com nutzen (Bezahlinformationen + Adresse)
	\item Vertraün durch gleiche "User Experience"
	\item leichtes einbinden in Website
	\end{itemize}

\subsubsection{ Kernkompetenzen}
\textbf{Differenzierung}\\
	\begin{itemize}
	\item Bezahlinformationen schon vorhanden
	\item vertrauter Ablauf
	\item Zusätzlich auch Adresse
	\end{itemize}
\textbf{Diversifikation}\\
	\begin{itemize}
	\item weite Verbreitung auf Webseiten
	\end{itemize}
\textbf{Imitationsschutz}\\
\textbf{Kundennutzen}\\
	\begin{itemize}
	\item Wenig aufwand
	\item Sicherheit, da keine neün Bezahldaten eingegen werden müssen
	\end{itemize}

\subsubsection{ Kennzahlen}
?



\section{Technologien und Zahlprozesse}

\subsection{ Einzahlung [8][.]}
	\begin{itemize}
	\item Kreditkarte, Debitkarte
	\item Lastschrift
	\item Pre-Paid (Guthabenbasiert - Konto aufladen)
	\end{itemize}


\subsection{ Bezahlmöglichkeiten}
\subsubsection{ Web}
\textbf{ virtülles Konto (payment) [3]}\\
	\begin{itemize}
	\item überweisungen im Vorraus (Pre-Paid)
	\item evtl. zustäzlich auch Lastschrift möglich
	\item überweisungen zwischen Usern möglich
	\end{itemize}

\textbf{ Checkout via Provider [9]}\\
	\begin{itemize}
	\item Webshop hat einen Provider z.B. Amazon Payments
	\item Provider übernimmt den Bezahlvorgang
	\item Provider hat Konto-/Kreditkartendaten hinterlegt 
	\end{itemize}

\textbf{ Kreditkarte}\\
	\begin{itemize}
	\item Website nimmt direkt Kreditkartendaten entgegen
	\item kein Dritter beteiligt, aber Kunde muss Kartendaten wieder jemandem abgeben.
	\end{itemize}

\textbf{ Email}\\
	\begin{itemize}
	\item Neuigkeit von Google wallet
	\item über Gmail geld versenden [2]
	\end{itemize}


\subsubsection{ Mobile [11]}
\textbf{ Mobile web payments (Desktop like) [6]}\\
	\begin{itemize}
	\item Einfache Adaption von ePayment auf mobile Geräte über mobiles Internet ist natürlich eine andere Möglichkeit, unterscheidet sich aber nicht wirklich vom oberen Punkt, daher nicht nochmals aufgeführt.
	\end{itemize}

\textbf{ premium SMS [4]}\\
	\begin{itemize}
	\item SMS Code an kostenpflichtige Nummer --> Bezahlen über Telefonrechnung
	\item in Asien und Europa verbreitet gewesen, wird nach und nach ersetzt
	\item Beispiel: 
	\subitem  Klingeltöne, 
	\subitem  Dial-a-coke von der Coca Cola Company
		\subitem  Ein Kunde kauft an einem Getränkeautomaten ein Erfrischungsgetränk und bezahlt es mit seinem mobilen Telefon. Dazu ruft er eine auf dem Automaten stehende Nummer an und wählt anschliessend an seinem mobilen Telefon ein Produkt aus. Das ausgewählte Produkt wird vom Automaten ausgegeben, die Bezahlung erfolgt über die Telefonrechnung. [11]
	\subitem  Touch \& Travel (Passt hier?)
		\subitem  Vodafone in Zusammenarbeit mit der Bahn AG
		\subitem  Erwerb des Tickets überflüssig macht
		\subitem  Bezahlen per Lastschrift
	\end{itemize}
		
		
\textbf{ Direct mobile billing [5]}\\
	\begin{itemize}
	\item Bezahlen über Handyrechnung (Netzanbieter)
	\item hohe Abgabenraten von 10-20\%
	\item Handynummer am auf Website angeben --> SMS mit Code --> diesen auf Website eingaben
	\end{itemize}


\textbf{ über NFC-Chip (google wallet, Touch\&Travel, girogo Sparkasse bis 20 Euro) [1][7]}\\
	\begin{itemize}
	\item Reichweite von ca. 10 cm. (gewünscht)
	\item deutscher Personalausweis 2011 ist NFC kompatibel
	\item zustätzlich PIN eingeben als sicherheit
	\item Andoid Secure Element API [12][13][14]
	\subitem  eigener Chip mit CPU/ ROM/ RAM/ I/O
	\subitem  sicheres Speichern von Daten (ausserhalb Main-OS)
	\subitem  über NFC lesbar oder interne API ansprechbar
	\item Sicherheitsgefahr z.B. durch kontaktieren aush geringer Enfernung (Vorüberlaufende Personen)
	\end{itemize}
	
\textbf{ (QR-Code) [10]}\\
	\begin{itemize}
	\item keine Echte Zahlmethode
	\item leitet auf Website, andere App um
	\item Vorteil: einfach, da nur Barcode-Scannen (NFC o.ä. nicht notwendig)
	\end{itemize}

%[.] Quelle notwendig
%[1] http://www.google.com/wallet/buy-in-store/
%[2] http://www.google.com/wallet/send-money/
%[3] http://paypal.com
%[4] http://en.wikipedia.org/wiki/Mobile_phone_micropayment#SMS.2FUSSD-based_transactional_payments
%[5] http://usatoday30.usatoday.com/tech/news/story/2012-04-04/mobile-billing-boku-zong/54003414/1
%[6] http://en.wikipedia.org/wiki/Mobile_phone_micropayment#Mobile_web_payments_.28WAP.29
%[7] http://en.wikipedia.org/wiki/Mobile_phone_micropayment#Contactless_Near_Field_Communication
%[8] http://en.wikipedia.org/wiki/Online_wallet
%[9] https://payments.amazon.com/sdui/sdui/business/cba
%[10]http://en.wikipedia.org/wiki/Mobile_phone_micropayment#QR_Code_Payments 
%[11]http://link.springer.com/chapter/10.1007/978-3-642-29802-8_9/fulltext.html#Sec15
%[12]http://link.springer.com/content/pdf/10.1007\%2F978-3-642-30436-1_1.pdf
%[13]https://code.google.com/p/seek-for-android/
%[14]http://nelenkov.blogspot.de/2012/08/accessing-embedded-secure-element-in.html
%[15]http://www.google.com/wallet/index.html
%[16]https://payments.amazon.com/sdui/sdui/home
%[17]http://support.google.com/wallet/bin/answer.py?hl=en\&answer=1347934
