\subsection{ Google Wallet [15]}
\subsubsection{ Das Unternehmen}
	\begin{itemize}
	\item Google Inc.
	\item präsentiert May 26, 2011 -- gestartet (app) September 19, 2011
	\end{itemize}

\subsubsection{ Geschäftsmodell}
\textbf{Funktionsweise}\\
	\begin{itemize}
	\item Kreditkarten/Debitkarten hinterlegt in Account
	\item Mobil (nur bei bestimmten Modellen möglich [17]) und im Web bezahlen möglich
	\item 24/7 fraud monitoring. Instant transaction notifications. 

	\item Sicherheit: User-Infos verschlüsselt, zusätzlicher PIN notwendig vor dem Bezahlen, muss PayPass-NFC-Reader berühren, über Website handy disable möglich
	\item Sicherheits-Risiko: möglichkeit des eavesdropping bei Google Analytics
	\end{itemize}


\textbf{Gewinne erwirtschafen}\\
	\begin{itemize}
	\item keine Abgaben sondern durch Werbung
	\item über Gmail Geld versenden: für wallet-account kostenlos, sonst 2,9\% abgabe
	\end{itemize}


\subsubsection{ Strategie}
	\begin{itemize}
	\item Zusammenlegung/Konzentation auf ein Dienst (Einstellung von Google-Checkout)
	\item NFC ausbreiten, Android als Plattform (Android Secure element[13])
	\item Android Market
	\item Macht der Größe des Unternehmens nutzen 
	\end{itemize}

\subsubsection{ Kernkompetenzen}
\textbf{Differenzierung}\\
	\begin{itemize}
	\item NFC
	\item Gmail geld per email
	\end{itemize}
\textbf{Diversifikation}\\
	\begin{itemize}
	\item breit gefächertes Angebot (Gmail, Checkout, NFC)
	\end{itemize}
\textbf{Imitationsschutz}\\
	\begin{itemize}
	\item im Web viel Konkurrenz daher:
	\item neü Bereiche wie NFC und Gmail-Geld
	\end{itemize}
\textbf{Kundennutzen}\\
	\begin{itemize}
	\item Wallet-Funktion: 
	\subitem  mehrere Karten hinterlegen
	\subitem  deaktivieren falls gestolen o.ä.
	\subitem  nachprüfen der Bezahlungen im Web
	\subitem  großer Anbieter --> viele Angebote (steigend)
	\end{itemize}

\subsubsection{ Kennzahlen}
	\begin{itemize}
	\item 300,000+ Mastercard-paypass Stores
	\item ca. 15 NFC-Handys und Tablets (Nexus, Samsung Galaxy S3-S4)[17]
	\end{itemize}




\subsection{ Amazon Payments [16]}
\subsubsection{ Das Unternehmen}
	\begin{itemize}
	\item Tochterfirma von Amazon.com 100\%
	\item 2007 gegründet
	\item alle Zahlinformationen der Mutterfirma, gleiche "checkout experience" wie auf Amazon.com
	\end{itemize}


\subsubsection{ Geschäftsmodell}
\textbf{Funktionsweise}\\
	\begin{itemize}
	\item "checkout by Amazon" (CBA)
	\subitem  direkt auf eigener Website einbinden nur Button (website nie verlassen)
	\subitem  Bezahlmethoden und Adressen von Amazon.com nutzen, keine Kreditkarte o.ä. wieder angeben
	\subitem  Anmeldung bei amazon über pop-up
	\item "Amazon Simple Pay"
	\subitem  ähnlich wie CBA
	\subitem  nur Bezahlprozess, keine Addressen oder sonstiges
	\end{itemize}
	
\textbf{Gewinne erwirtschafen}\\
	\begin{itemize}
	\item Transaktionsabgaben: 2.9\% + \$0.30
	\item Rabatt für große monatliche Raten. Bei 100.000\$ -> nur noch 1.9\%
	\item besonderes für Micropayment: 5\% + \$0.05
	\item besonderes für Non-profit: 2.2\% + \$0.30
	\end{itemize}


\subsubsection{ Strategie}
	\begin{itemize}
	\item große Kundenbasis von amazon.com nutzen (Bezahlinformationen + Adresse)
	\item Vertraün durch gleiche "User Experience"
	\item leichtes einbinden in Website
	\end{itemize}

\subsubsection{ Kernkompetenzen}
\textbf{Differenzierung}\\
	\begin{itemize}
	\item Bezahlinformationen schon vorhanden
	\item vertrauter Ablauf
	\item Zusätzlich auch Adresse
	\end{itemize}
\textbf{Diversifikation}\\
	\begin{itemize}
	\item weite Verbreitung auf Webseiten
	\end{itemize}
\textbf{Imitationsschutz}\\
\textbf{Kundennutzen}\\
	\begin{itemize}
	\item Wenig aufwand
	\item Sicherheit, da keine neün Bezahldaten eingegen werden müssen
	\end{itemize}

\subsubsection{ Kennzahlen}
?
