\subsection{ Google Wallet [15]}
\subsubsection{ Das Unternehmen}
Google Wallet ist ein Online-Bezahlsystem des Unternehmens Google Inc. Vorgestellt im Mai 2011 startete der Dienst offiziell im September 2011.
Damit stellt Google Wallet außerdem auch den Nachfolger von Google Checkout dar, welcher nach Übergangszeit nach und nach ersetzt wird.

\subsubsection{ Geschäftsmodell}
\textbf{Funktionsweise}\\
Wie der Name aussagt stellt Google Wallet eine "Brieftasche" dar in mehrerer verschiende Kredit- oder Debitkarten hinterlegt werden können.
Als eins der Hauptmerkmal des Dienstes steht das mobile Bezahlen durch ein Android-Smartphone mit NFC-Chip da. Dies ist jedoch vererst nur bei bestimmten Handymodellen möglich. \footnote{\url{http://support.google.com/wallet/bin/answer.py?hl=en\&answer=1347934}}
Geworben wird mit der Sicherheit sowohl für den Kunden als auch für den Händler. Durch ständige Betrugsüberwachung ("24/7 fraud monitoring") um jederzeit ungewöhnliche Aktivitäten festzustellen. Weitere Sicherheit für den Kunden soll es dadurch geben, dass alle Daten verschlüsselt gespeichert und übertragen werden. Um mobil zu Bezahlen muss sich das Handy Zentimeter vom PayPass-NFC-Reader befinden und zusätzlich über einen PIN bestätigt werden.\\
Dennoch gibt es auch kritische Stimmen. So wurde z.B. die Möglichkeit das sogenannte "eavesdropping" über die Verwendung von Google Analytics kritisiert, womit es möglich sein kann PINs auszulesen.  \footnote{\url{http://www.itworld.com/mobile-wireless/248596/even-after-rewrites-google-wallet-retains-gaping-security-holes-mainly-due-an}}


\textbf{Gewinne erwirtschafen}\\
Derzeit belastet Google Wallet den Benutzer mit keinerlei Gebühren oder Transaktionsabgaben. Angeleht an das Geschäftsmodell vieler Google-Prokukte wird mithilfe von Werbung Gewinn erwirtschaftet.
Ausnahme ist hierbei der spezielle Email-Dienst GMail, über den Wallet-Benutzer Geld per Email versenden können. Hier wird für Addressaten ohne Wallet Account eine Abgabe von 2,9% berechnet.

	
\subsubsection{ Strategie}
Google Inc. versucht mit Google Wallet den Zusammenschluss verschiedener Bezahlmöglichkeiten, daher wurde auch Google Checkout ersetzt.
Mit Hilfe der großen Marktmacht des Unternehmen, unter Anderem auch im mobilen Markt mit dem Betriebsystem Android, versucht es den Dienst populärer zu machen. Dabei spielt wie schon erwähnt das mobile Betriebsystem Andoid - mit einem derzeitigen Marktanteil unter Smartphones von ca. 75% - eine große Rolle. Hier ist es im Interesse von Google die Nahfunk-Technologie NFC weiter zu verbreiten. So wurde extra für solche Anwendungen mit dem "Android Secure Element" \footnote{\url{https://code.google.com/p/seek-for-android/ neues Modul}} ein neues Modul zum sicheren Speichern von sensiblen Informationen entwickelt.


\subsubsection{ Kernkompetenzen}
\textbf{Differenzierung}\\
Im mobilen Bereich mit der Verwendung der Nahfunk-Technologie NFC ist Google Wallet einer der frühen und wenigen Anbieter.\\
Auch das Senden von Geld per Email über einen GMail-Account stellt ein Novum dar.

\textbf{Diversifikation}\\
Typisch für Google ist eine Fächerung des Angebots. So kann Wallet "ganz normal" online als Webanwendung genutzt werden, mobil ist es durch eine Andoid-App im Einsatz. Mit dem Versenden von Geld per Email bietet Google nochmals eine neue Möglichkeit des ePayments an.

\textbf{Imitationsschutz}\\
Im Bereich Web gibt es viel Konkurrenz daher wird besonders der mobile Bereich verstärkt fokussiert. Besonders eine Anbindung an den Android Playstore ist natürlich hilfreich.

\textbf{Kundennutzen}\\
Beworben wird vor allem der Nutzen als "Brieftasche", durch das Hinterlegen von verschiedenen Kreditkarten. So ist auch ein ferngesteuertes Deaktivieren des Accounts möglich, falls z.B. das Smartphone verloren oder gestohlen wurde. Auch eine Auflistung aller Bezahlungen ist über eine Webanwendung möglich.\\
Auch die immer größer werdende Verbreitung des Dienstes, sowie die intergrierung in den Android Playstore führt zu immer mehr Angeboten.

\subsubsection{ Kennzahlen}
Mobiles Bezahlen ist derzeit nur in den Vereinigten Staaten von Amerika möglich. Hier gibt es inzwischen über 300.000 "Mastercard-paypass Stores", welche Google Wallet als Zahlungsart aktzeptieren.
Derzeit ist die App auf 15 Smartphone und Tabletmodellen möglich. Dazu gehören neben der Google-eigenen "Nexus-Reihe" z.B. auch das Samsung Galasy S3 oder S4. 


\subsection{ Amazon Payments [16]}
\subsubsection{ Das Unternehmen}
	\begin{itemize}
	\item Tochterfirma von Amazon.com 100\%
	\item 2007 gegründet
	\item alle Zahlinformationen der Mutterfirma, gleiche "checkout experience" wie auf Amazon.com
	\end{itemize}


\subsubsection{ Geschäftsmodell}
\textbf{Funktionsweise}\\
	\begin{itemize}
	\item "checkout by Amazon" (CBA)
	\subitem  direkt auf eigener Website einbinden nur Button (website nie verlassen)
	\subitem  Bezahlmethoden und Adressen von Amazon.com nutzen, keine Kreditkarte o.ä. wieder angeben
	\subitem  Anmeldung bei amazon über pop-up
	\item "Amazon Simple Pay"
	\subitem  ähnlich wie CBA
	\subitem  nur Bezahlprozess, keine Addressen oder sonstiges
	\end{itemize}
	
\textbf{Gewinne erwirtschafen}\\
	\begin{itemize}
	\item Transaktionsabgaben: 2.9\% + \$0.30
	\item Rabatt für große monatliche Raten. Bei 100.000\$ -> nur noch 1.9\%
	\item besonderes für Micropayment: 5\% + \$0.05
	\item besonderes für Non-profit: 2.2\% + \$0.30
	\end{itemize}


\subsubsection{ Strategie}
	\begin{itemize}
	\item große Kundenbasis von amazon.com nutzen (Bezahlinformationen + Adresse)
	\item Vertraün durch gleiche "User Experience"
	\item leichtes einbinden in Website
	\end{itemize}

\subsubsection{ Kernkompetenzen}
\textbf{Differenzierung}\\
	\begin{itemize}
	\item Bezahlinformationen schon vorhanden
	\item vertrauter Ablauf
	\item Zusätzlich auch Adresse
	\end{itemize}
\textbf{Diversifikation}\\
	\begin{itemize}
	\item weite Verbreitung auf Webseiten
	\end{itemize}
\textbf{Imitationsschutz}\\
\textbf{Kundennutzen}\\
	\begin{itemize}
	\item Wenig aufwand
	\item Sicherheit, da keine neün Bezahldaten eingegen werden müssen
	\end{itemize}

\subsubsection{ Kennzahlen}
?
