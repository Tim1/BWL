\subsection[Bitcoin]{Bitcoin\footnote{\url{http://bitcoin.org/}}}
\subsubsection{Die Hintergründe}
\begin{itemize}
	\item kein Unternehmen, eher Konzept/Software
	\item Erfinder/Ersteller unbekannt (Pseudonym "Satoshi Nakamoto")
	\item 1998: Konzept "crypto-currency" erstmals beschrieben von Wei Dai in der cypherpunks Mailing-List
	\item 2009: Bitcoin Spezifikation und Proof of Concept von "Satoshi Nakamoto" in einer Kryprographie Mailing-List veröffentlicht
	\item 2012/09/27: gründung der Bitcoin Foundation\footnote{\url{https://bitcoinfoundation.org/}}
\end{itemize}
\subsubsection{Geschäftsmodell}
\textbf{Funktionsweise}
\begin{itemize}
	\item Um Bitcoins zu erlangen müssen entweder neue errechnet oder bereits errechnete erworben werden
	\item Sicherheit:
	\begin{itemize}
		\item Einzigartigkeit von Transaktionen gewährleistet durch die Verwendung einer Block Chain\footnote{\url{https://en.bitcoin.it/wiki/Block_chain}}
		\item Auf Identität der an einer Transaktion beteiligten Personen kann nicht anhand derer Bitcoin-Adresse geschlossen werden
		\item Für jede Transaktion kann eine neue Bitcoin-Adresse verwendet werden, um "Profiling" einer Bitcoin-Adresse durchzuführen
	\end{itemize}
	\item Sicherheits-Risiko:
	\begin{itemize}
		\item alle Bitcoin Transaktionen sind öffentlich und permanent im Netzwerk gespeichert
		\item Verlust des Walltes bedeutet umgehenden Verlust des Geldes
	\end{itemize}
\end{itemize}
\textbf{Gewinne erwirtschaften}
\begin{itemize}
	\item Freiwillige Transaktionsgebühr (ausgezahlt an denjenigen, der die Transaktion bestätigt\footnote{Bestätigen von Transaktionen = aufwändig (Rechenzeit/-leistung), Transaktionsgebühr bietet anderen Nodes des Bitcoin-Netzwerks also Anreiz, die Transaktion zu bestätigen})
\end{itemize}

\subsubsection{Strategie}
\begin{itemize}
	\item Schwer zu sagen da keine Person/Institution direkt dahinter
	\item Strategie/Vision der Bitcoin Foundation: Standardisierung, Gewährleistung von Sicherheit, Promotion
\end{itemize}

\subsubsection{Kernkompetenzen}
\begin{itemize}
	\item Sehr hoher Grad an Anonymität
	\item Sehr geringe Kosten für den "Dienst" an sich (Transaktionsgebühren theoretisch vollkommen freiwillig)
\end{itemize}

\subsubsection{Kennzahlen}
\begin{itemize}
	\item Täglicher Handel im Wert von Millionen USD verteilt auf ~50.000 Transaktionen
	\item Wert der Bitcoins im Umlauf USD 1,3 Mrd.
	\item Anzahl der Bitcoins fest limitiert auf 21 Mio.
	\item Bitcoins sind teilbar auf bis zu 8 Dezimalstellen, daher ~21*10\textasciicircum 14 Währungseinheiten.
\end{itemize}
