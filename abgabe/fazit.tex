\clearpage
\section{Fazit}
- rießiger markt, verschiedene möglichkeiten
- mobil immer wichtiger,
	- jeder hat handy potenzial, Marktdurchdringung
	- internetfähiger Mobiltelefone
	- technische Fortschritt
- Vertrauen in Sicherheit, 


Es existiert eine Vielzahl von Möglichkeiten, seine Waren im Internet zu bezahlen.
Vorhersagen, welche Verfahren mit welchen Betreibern sich durchsetzen werden, sind
sicherlich sehr schwer bis unmöglich. Noch vor einiger Zeit ist dem CyberCash-
Verfahren eine gute Zukunft vorhergesagt worden. Das System ist inzwischen
eingestellt worden, ebenso eCash der Deutschen Bank24. Somit scheint eine Etablier-
ung des Cybergeldes mehr als fraglich.
Mit entscheidend für die Verbreitung eines neuen Systems ist der Einstiegsaufwand
und die schnelle Nutzbarkeit des Systems. Dies haben beispielsweise Firstgate und
Paybox gut erkannt. Sofort nach der Anmeldung stehen die Systeme zur Verfügung.
Hierdurch entstehen zwar Mißbrauchsmöglichkeiten, die durch die festgelegten Limits
jedoch eingeschränkt werden.
Eine schnelle anfängliche Verbreitung eines neuen Systems ist ebenfalls von großer
Bedeutung, denn sonst warten die Kunden auf die Händler und umgekehrt. Der
Akzeptanz- und Verbreitungsgrad kann beispielsweise durch Maßnahmen erhöht
werden, wie sie Visa für SET plant.
Sicherheit ist für die meisten Online-Shopper das wichtigste Kriterium bei der
Auswahl eines Zahlungsmittels [KAR01]. Trotzdem ist fast keiner bereit, für die
Sicherheit mehr Geld auszugeben, beispielsweise für die Anschaffung eines teuren
Kartenlesegerätes.
Zusätzlich zu den angesprochenen Verfahren gibt es für Shops noch die Möglichkeit,
die Zahlungen über Transaktionsdienstleister abzuwickeln. Diese kümmern sich um
die Einbindung neuer Verfahren. In nächster Zeit soll es auch vorausbezahlte Karten,
die am Kiosk oder an der Tankstelle gekauft werden können, in Deutschland geben.
Diese funktionieren ähnlich wie die Prepaid-Karten im Mobilfunkbereich.
Im Laufe der Zeit werden sich sicherlich einige Systeme durchsetzen, andere werden
eher Nischenplätze einnehmen oder eingestellt. Welche Systeme welche Rolle spielen
werden, ist nicht vorhersehbar.

Fast jeder Bundesbürger besitzt heute
ein Mobiltelefon und hat es so gut
wie immer dabei. Die zukünftige Be-
deutung von M-Payment sollte wegen
der hohen Marktdurchdringung von
Mobiltelefonen und der wachsenden
Bedeutung des M-Commerce nicht
unterschätzt werden. Die einzelnen
Mobile Payment-Verfahren sind vielen
Befragten schon bekannt und auch
die Bereitschaft, sie in Zukunft zu nut
-
zen, zeigt weiteres Wachstumspoten-
zial auf. Der technische Fortschritt
tut darüber hinaus sein Übriges: Die
Anzahl internetfähiger Mobiltelefone
sowie die mobile Internetnutzung
sind bereits hoch und steigen laut
Aussagen von Experten weiter an.
Die vorliegende Studie zeigt, dass
diese Entwicklung auch ein Treiber
für M-Payment sein kann, da insbe
-
sondere die Smartphone-Nutzer und
Intensivnutzer des mobilen Internets
die technischen Voraussetzungen
für M-Payment bereits erfüllen. Sie
haben ein Interesse daran, mobile
Zahlungen zu tätigen und sehen we-
niger Barrieren für die Nutzung von
Mobile Payment