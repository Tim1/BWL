\section{Technologien und Zahlprozesse}

\subsection{Einzahlung}
Für alle Online-Bezahldienste muss natürlich eine Geldquelle hinterlegt sein. Das Bezahlen kann über eine der drei Möglichkeiten geschehen: "Pre-Paid", "Pay now", "Pay later".\\
Guthabenbasiertes (Pre-Paid) findet man z.B. bei Paypal. Hier kann per Überweisung auf ein virtuelles Konto eingezahlt werden. Nach Geldeingang kann nun über das Geld in diesem Konto verfügt und Bezahlungen getätigt werden.
Ebenfalls sehr verbreitet (vor Allem in den USA) ist das Hinterlegen von Kreditkarten oder Debitkarten. Eine weitere Möglichkeit ist das Hinterlegen der Kontodaten und Bezahlen über Lastschrift.\footnote{\url{http://en.wikipedia.org/wiki/Online_wallet}}

\subsection{Bezahlmöglichkeiten}
\subsubsection{Web}
\textbf{Virtulles Konto}\\
Ein virtuelles oder Online-Konto fungiert als Zwischenglied zwischen Kundenkonto und Verkäufer. Oftmals kann der Kunde von seinem Konto aus im Vorraus Überweisungen tätigen (Pre-Paid), welche dann auf diesem Konto gutgeschrieben sind. Auch kann über Lastschrift oder Hinterlegen von Kreditkarteninformationen zustätzlich Zahlungsmöglichkeiten geschaffen werden.\\
Überweisungen zwischen verschieden Kundenkonten ist oftmals auch möglich.\\
Als Beispiel hierfür ist u.A. Paypal zu nennen.\footnote{\url{http://paypal.com}}

\textbf{Checkout via Provider}\\
Der vermutlich "gewöhnlichste" Bezahlvorgang. Hierbei wird beim Provider Konto-/Kreditkartendaten hinterlegt, diese sind dann dort gespeichert und werden für Abbuchungen verwendet.
Der jeweilige Provider übernimmt dabei den gesammten Bezahlvorgang. Ein Bezahlvorgang sieht meist folgendermaßen aus: Kunde legt in einem Webshop einen Artikel in den Warenkorb und wählt die Bezahlung über den jeweiligen Provider aus, somit ist der Vorgang für ihn abgeschlossen. Der Verkäufer erhält schließlich das Geld vom Provider, abzüglich der Transaktionskosten.\\
Beispiele hierfür ist u.A. Amazon Payments.\footnote{\url{https://payments.amazon.com/sdui/sdui/business/cba}}\\

\textbf{Kreditkarte}\\
Ebenfalls möglich ist auch die direkte Bezahlmöglichkeit über Kreditkarte. Hierbei muss der Kunde seine Kreditkartendaten direkt an den Verkäufer übermitteln. Dabei ist neben dem Kreditkarteninstitut kein weiterer Partner beteiligt. Hierbei ist jedoch für den Kunden problematisch seine Kartendaten einer weiteren Partei zur Verfügung zu stellen. Für den Verkäufer gibt es eventuell die Gefahr von nicht gedeckten Karten.\\

\textbf{Email}\\
Neu von Google Wallet eingeführt ist das Versenden von Geld via Email. Dabei muss der Absender einen Account haben und kann damit über sein GMail-Konto an beliebgige Personen Geld versenden. Der Empfänger bekommt das Geld ebenfalls auf ein Wallet-Account übertragen. Hat er diesen nicht, so muss er sich erst registrieren um über das Geld verfügen zu können.\footnote{\url{http://www.google.com/wallet/send-money/}}\\

\subsubsection{Mobile}
\textbf{Mobile Web Payments\footnote{\url{http://en.wikipedia.org/wiki/Mobile_phone_micropayment\#Mobile_web_payments_.28WAP.29}}}\\
Eine einfache Möglichkeit mobile Bezahlen zu können, ist die Adaption von ePayment-Webanwendung für mobile Devices. Über WLAN oder mobiles Internet kann die Verbindung zu einem Webserver hergestellt werden, die Vorgänge gleichen dabei den Abläufen einer Desktop-Anwendung.\footnote{\url{http://link.springer.com/chapter/10.1007/978-3-642-29802-8_9/fulltext.html\#Sec15}}\\

\textbf{Premium SMS\footnote{\url{http://en.wikipedia.org/wiki/Mobile_phone_micropayment\#SMS.2FUSSD-based_transactional_payments}}}\\
Einer der ältesten mobilen Bezahlmöglichkeit sind die sogenannten Premium SMS. Über das Senden von SMS an kostenpflichtige Nummern kann ebenfalls Geld transferiert werden. Dabei bezahlt der Kunde über die Telefonrechnung. Dabei spielt der Netzprovider die Vermittlungsrolle zwischen Kunde und Verkäufer.\\
Besonders in Europa und Asien war und ist diese Variante des mobilen Bezahlens weit verbreitet, wird jedoch nach und nach durch neuere Möglichkeiten ersetzt. Vor allem die hohen Abgaben an den Mobilfunkprovider von bis zu 40-50\% ist ein Grund hierfür.\\
Verwendet wird es vor allem für sehr kleine Beiträge (Micropayment) wie z.B. für Klingeltöne, Logos, TV-Votings. Es gab jedoch auch einige ausgefallenere Beispiele, wie etwa das Program "Dial-a-coke" von der Coca Cola Company: Ein Kunde kauft an einem Getränkeautomaten ein Erfrischungsgetränk und bezahlt es mit seinem mobilen Telefon. Dazu ruft er eine auf dem Automaten stehende Nummer an und wählt anschliessend an seinem mobilen Telefon ein Produkt aus. Das ausgewählte Produkt wird vom Automaten ausgegeben, die Bezahlung erfolgt über die Telefonrechnung.\footnote{\url{http://link.springer.com/chapter/10.1007/978-3-642-29802-8_9/fulltext.html\#Sec15}}\\
 
\textbf{Direct mobile billing\footnote{\url{http://usatoday30.usatoday.com/tech/news/story/2012-04-04/mobile-billing-boku-zong/54003414/1}}}\\
Verwandt mit dem Premium SMS verwendet auch "Direct mobile billing" den Mobilfunkanbieter als Zwischenhändler.\\
Typischerweise gibt hier der Kunde dem Verkäufer seine Handynummer an und erhält dann nach kurzer Zeit eine SMS mit einem bestimmten Code. Diesen kann er schließlich auf einer Webseite oder einer App eingeben um so den Kaufvorgang abzuschließen. Belastet wird hierbei wieder die Telefonrechnung.
Trotz hohen Abgaben von 10-20\% an den Netzanbieter steiget der Umsatz dieser Bezahlmethode aufgrund der Einfachheit in den letzten Jahren weiter.\\


\textbf{NFC}\\
Near Field Communication\footnote{\url{https://de.wikipedia.org/wiki/Near_Field_Communication}}, kurz NFC ist eine Technik zum kontaktlosen Austausch von Daten über sehr kurze Strecken, typischweise von wenigen Zentimetern. Diese kurze Reichweite ist erwünscht um den Kunden mehr Sicherheit zu bieten. So muss das Smartphone mit NFC-Chip oftmals direkt auf das Lesegerät gelegt bzw. daran gehalten werden. \\
Da die Technologie noch relativ neu ist, gibt es vorerst nur wenige Geräte mit NFC-Unterstützung. Doch besonders mit dem mobilen Betriebsystem Android scheint die Technologie langsam Fuß zu fassen. So wurde mit dem "Android Secure Element" und seiner API\footnote{\url{https://code.google.com/p/seek-for-android/}} ein eigenes Modul mitsamt Chipsatz und externen Speicher geschaffen um das sichere Speichern von sensiblen Daten, wie sie beim NFC anfallen, zu ermöglichen.\footnote{\url{http://link.springer.com/content/pdf/10.1007\%2F978-3-642-30436-1_1.pdf}}\\
Anbieter welche auf auf diese Technologie setzten sind neben schon erwähnten Google Wallet auch die von Deutscher Bahn und Vodafone geschaffene Programm Touch\&Travel, oder die Sparkasse girogo, welche damit Zahlungen bis 20€ anbietet.
Seit 2011 ist auch der deutsche Personalausweis NFC-kompatibel.\footnote{\url{https://de.wikipedia.org/wiki/Near_Field_Communication\#Echteinsatz}}\\
Gefährlich - natürlich neben dem Verlust des NFC-Gerätes - kann das beiläufige Auslesen von Daten sein. Dies und der Schutz der Privatsphäre kann nur dann sichergestellt werden, wenn die Funktionen im Normalfall komplett abgeschaltet sind.\\

\textbf{QR-Code\footnote{\url{http://en.wikipedia.org/wiki/Mobile_phone_micropayment\#QR_Code_Payments}}}\\
QR-Codes erfreuen sich seit einiger Zeit großer Beliebtheit zur Übertragen von kurzen Informationen auf ein mobiles Gerät. Die weite Verbreitung und Einfachheit des Scannvorgangs findet auch im mobilen Bezahlsystemen ihren Einsatz. So kann beispielsweise an der Kunde z.B. an der Kasse einen Code für seinen Einkauf scannen und hat sogleich die Höhe des Betrages auf seinem mobilen Gerät. Nun könnte er entweder auf eine Webseite geleitet werden um den Betrag zu bezahlen oder direkt mit einer App des Verkäufers zum Bezahlen aufgefordert werden.\\
Damit stellt es keine eigenständige Bezahlform dar, doch durch die Einfachheit des Scannvorgangs könnte sich das Verfahren noch weiter verbreiten. 