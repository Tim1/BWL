\section{Technologien und Zahlprozesse}

\subsection{Einzahlung [8][.]}
Für alle Online-Bezahldienste muss natürlich eine Geldquelle hinterlegt sein. Das Bezahlen kann über eine der drei Möglichkeiten geschehen: "Pre-Paid", "Pay now", "Pay later".\\
Guthabenbasiertes (Pre-Paid) findet man z.B. bei Paypal. Hier kann per Überweisung auf ein virtuelles Konto eingezahlt werden. Nach Geldeingang kann nun über das Geld in diesem Konto verfügt und Bezahlungen getätigt werden.
Ebenfalls sehr verbreitet (vor Allem in den USA) ist das Hinterlegen von Kreditkarten oder Debitkarten. Eine weitere Möglichkeit ist das Hinterlegen der Kontodaten und Bezahlen über Lastschrift.

\subsection{Bezahlmöglichkeiten}
\subsubsection{Web}
\textbf{Virtulles Konto (payment) [3]}\\


        \begin{itemize}
        \item überweisungen im Vorraus (Pre-Paid)
        \item evtl. zustäzlich auch Lastschrift möglich
        \item überweisungen zwischen Usern möglich
        \end{itemize}

\textbf{ Checkout via Provider [9]}\\
        \begin{itemize}
        \item Webshop hat einen Provider z.B. Amazon Payments
        \item Provider übernimmt den Bezahlvorgang
        \item Provider hat Konto-/Kreditkartendaten hinterlegt
        \end{itemize}

\textbf{ Kreditkarte}\\
        \begin{itemize}
        \item Website nimmt direkt Kreditkartendaten entgegen
        \item kein Dritter beteiligt, aber Kunde muss Kartendaten wieder jemandem abgeben.
        \end{itemize}

\textbf{ Email}\\
        \begin{itemize}
        \item Neuigkeit von Google wallet
        \item über Gmail geld versenden [2]
        \end{itemize}


\subsubsection{ Mobile [11]}
\textbf{ Mobile web payments (Desktop like) [6]}\\
        \begin{itemize}
        \item Einfache Adaption von ePayment auf mobile Geräte über mobiles Internet ist natürlich eine andere Möglichkeit, unterscheidet sich aber nicht wirklich vom oberen Punkt, daher nicht nochmals aufgeführt.
        \end{itemize}

\textbf{ premium SMS [4]}\\
        \begin{itemize}
        \item SMS Code an kostenpflichtige Nummer --> Bezahlen über Telefonrechnung
        \item in Asien und Europa verbreitet gewesen, wird nach und nach ersetzt
        \item Beispiel:
        \subitem  Klingeltöne,
        \subitem  Dial-a-coke von der Coca Cola Company
ubitem  Ein Kunde kauft an einem Getränkeautomaten ein Erfrischungsgetränk und bezahlt es mit seinem mobilen Telefon. Dazu ruft er eine auf dem Automaten stehende Nummer an und wählt anschliessend an seinem mobilen Telefon ein Produkt aus. Das ausgewählte Produkt wird vom Automaten ausgegeben, die Bezahlung erfolgt über die Telefonrechnung. [11]
        \subitem  Touch \& Travel (Passt hier?)
                \subitem  Vodafone in Zusammenarbeit mit der Bahn AG
                \subitem  Erwerb des Tickets überflüssig macht
                \subitem  Bezahlen per Lastschrift
        \end{itemize}


\textbf{ Direct mobile billing [5]}\\
        \begin{itemize}
        \item Bezahlen über Handyrechnung (Netzanbieter)
        \item hohe Abgabenraten von 10-20\%
        \item Handynummer am auf Website angeben --> SMS mit Code --> diesen auf Website eingaben
        \end{itemize}


\textbf{ über NFC-Chip (google wallet, Touch\&Travel, girogo Sparkasse bis 20 Euro) [1][7]}\\
        \begin{itemize}
        \item Reichweite von ca. 10 cm. (gewünscht)
        \item deutscher Personalausweis 2011 ist NFC kompatibel
        \item zustätzlich PIN eingeben als sicherheit
        \item Andoid Secure Element API [12][13][14]
        \subitem  eigener Chip mit CPU/ ROM/ RAM/ I/O
        \subitem  sicheres Speichern von Daten (ausserhalb Main-OS)
        \subitem  über NFC lesbar oder interne API ansprechbar
        \item Sicherheitsgefahr z.B. durch kontaktieren aush geringer Enfernung (Vorüberlaufende Personen)
        \end{itemize}

\textbf{ (QR-Code) [10]}\\
        \begin{itemize}
        \item keine Echte Zahlmethode
        \item leitet auf Website, andere App um
        \item Vorteil: einfach, da nur Barcode-Scannen (NFC o.ä. nicht notwendig)
        \end{itemize}

%[.] Quelle notwendig
%[1] http://www.google.com/wallet/buy-in-store/
%[2] http://www.google.com/wallet/send-money/
%[3] http://paypal.com
%[4] http://en.wikipedia.org/wiki/Mobile_phone_micropayment#SMS.2FUSSD-based_transactional_payments
%[5] http://usatoday30.usatoday.com/tech/news/story/2012-04-04/mobile-billing-boku-zong/54003414/1
%[6] http://en.wikipedia.org/wiki/Mobile_phone_micropayment#Mobile_web_payments_.28WAP.29
%[7] http://en.wikipedia.org/wiki/Mobile_phone_micropayment#Contactless_Near_Field_Communication
%[8] http://en.wikipedia.org/wiki/Online_wallet
%[9] https://payments.amazon.com/sdui/sdui/business/cba
%[10]http://en.wikipedia.org/wiki/Mobile_phone_micropayment#QR_Code_Payments 
%[11]http://link.springer.com/chapter/10.1007/978-3-642-29802-8_9/fulltext.html#Sec15
%[12]http://link.springer.com/content/pdf/10.1007\%2F978-3-642-30436-1_1.pdf
%[13]https://code.google.com/p/seek-for-android/
%[14]http://nelenkov.blogspot.de/2012/08/accessing-embedded-secure-element-in.html
%[15]http://www.google.com/wallet/index.html
%[16]https://payments.amazon.com/sdui/sdui/home
%[17]http://support.google.com/wallet/bin/answer.py?hl=en\&answer=1347934
