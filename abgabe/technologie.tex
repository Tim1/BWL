\section{Technologien und Zahlprozesse}

\subsection{ Einzahlung [8][.]}
	\begin{itemize}
	\item Kreditkarte, Debitkarte
	\item Lastschrift
	\item Pre-Paid (Guthabenbasiert - Konto aufladen)
	\end{itemize}


\subsection{ Bezahlm�glichkeiten}
\subsubsection{ Web}
\textbf{ virt�lles Konto (payment) [3]}\\
	\begin{itemize}
	\item �berweisungen im Vorraus (Pre-Paid)
	\item evtl. zust�zlich auch Lastschrift m�glich
	\item �berweisungen zwischen Usern m�glich
	\end{itemize}

\textbf{ Checkout via Provider [9]}\\
	\begin{itemize}
	\item Webshop hat einen Provider z.B. Amazon Payments
	\item Provider �bernimmt den Bezahlvorgang
	\item Provider hat Konto-/Kreditkartendaten hinterlegt 
	\end{itemize}

\textbf{ Kreditkarte}\\
	\begin{itemize}
	\item Website nimmt direkt Kreditkartendaten entgegen
	\item kein Dritter beteiligt, aber Kunde muss Kartendaten wieder jemandem abgeben.
	\end{itemize}

\textbf{ Email}\\
	\begin{itemize}
	\item Neuigkeit von Google wallet
	\item �ber Gmail geld versenden [2]
	\end{itemize}


\subsubsection{ Mobile [11]}
\textbf{ Mobile web payments (Desktop like) [6]}\\
	\begin{itemize}
	\item Einfache Adaption von ePayment auf mobile Ger�te �ber mobiles Internet ist nat�rlich eine andere M�glichkeit, unterscheidet sich aber nicht wirklich vom oberen Punkt, daher nicht nochmals aufgef�hrt.
	\end{itemize}

\textbf{ premium SMS [4]}\\
	\begin{itemize}
	\item SMS Code an kostenpflichtige Nummer --> Bezahlen �ber Telefonrechnung
	\item in Asien und Europa verbreitet gewesen, wird nach und nach ersetzt
	\item Beispiel: 
	\subitem  Klingelt�ne, 
	\subitem  Dial-a-coke von der Coca Cola Company
		\subitem  Ein Kunde kauft an einem Getr�nkeautomaten ein Erfrischungsgetr�nk und bezahlt es mit seinem mobilen Telefon. Dazu ruft er eine auf dem Automaten stehende Nummer an und w�hlt anschliessend an seinem mobilen Telefon ein Produkt aus. Das ausgew�hlte Produkt wird vom Automaten ausgegeben, die Bezahlung erfolgt �ber die Telefonrechnung. [11]
	\subitem  Touch \& Travel (Passt hier?)
		\subitem  Vodafone in Zusammenarbeit mit der Bahn AG
		\subitem  Erwerb des Tickets �berfl�ssig macht
		\subitem  Bezahlen per Lastschrift
	\end{itemize}
		
		
\textbf{ Direct mobile billing [5]}\\
	\begin{itemize}
	\item Bezahlen �ber Handyrechnung (Netzanbieter)
	\item hohe Abgabenraten von 10-20\%
	\item Handynummer am auf Website angeben --> SMS mit Code --> diesen auf Website eingaben
	\end{itemize}


\textbf{ �ber NFC-Chip (google wallet, Touch\&Travel, girogo Sparkasse bis 20 Euro) [1][7]}\\
	\begin{itemize}
	\item Reichweite von ca. 10 cm. (gew�nscht)
	\item deutscher Personalausweis 2011 ist NFC kompatibel
	\item zust�tzlich PIN eingeben als sicherheit
	\item Andoid Secure Element API [12][13][14]
	\subitem  eigener Chip mit CPU/ ROM/ RAM/ I/O
	\subitem  sicheres Speichern von Daten (ausserhalb Main-OS)
	\subitem  �ber NFC lesbar oder interne API ansprechbar
	\item Sicherheitsgefahr z.B. durch kontaktieren aush geringer Enfernung (Vor�berlaufende Personen)
	\end{itemize}
	
\textbf{ (QR-Code) [10]}\\
	\begin{itemize}
	\item keine Echte Zahlmethode
	\item leitet auf Website, andere App um
	\item Vorteil: einfach, da nur Barcode-Scannen (NFC o.�. nicht notwendig)
	\end{itemize}

%[.] Quelle notwendig
%[1] http://www.google.com/wallet/buy-in-store/
%[2] http://www.google.com/wallet/send-money/
%[3] http://paypal.com
%[4] http://en.wikipedia.org/wiki/Mobile_phone_micropayment#SMS.2FUSSD-based_transactional_payments
%[5] http://usatoday30.usatoday.com/tech/news/story/2012-04-04/mobile-billing-boku-zong/54003414/1
%[6] http://en.wikipedia.org/wiki/Mobile_phone_micropayment#Mobile_web_payments_.28WAP.29
%[7] http://en.wikipedia.org/wiki/Mobile_phone_micropayment#Contactless_Near_Field_Communication
%[8] http://en.wikipedia.org/wiki/Online_wallet
%[9] https://payments.amazon.com/sdui/sdui/business/cba
%[10]http://en.wikipedia.org/wiki/Mobile_phone_micropayment#QR_Code_Payments 
%[11]http://link.springer.com/chapter/10.1007/978-3-642-29802-8_9/fulltext.html#Sec15
%[12]http://link.springer.com/content/pdf/10.1007\%2F978-3-642-30436-1_1.pdf
%[13]https://code.google.com/p/seek-for-android/
%[14]http://nelenkov.blogspot.de/2012/08/accessing-embedded-secure-element-in.html
%[15]http://www.google.com/wallet/index.html
%[16]https://payments.amazon.com/sdui/sdui/home
%[17]http://support.google.com/wallet/bin/answer.py?hl=en\&answer=1347934
