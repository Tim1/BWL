\subsection{Google Wallet}
\subsubsection{Das Unternehmen}

\textbf{Eckdaten}\\
- Neuer Anbieter\\
- 1425 Verkaufsstellen\\
- nur in Deutschland vertreten\\
- 190 Partner im Internet\\
\\
%
% für neue kleine Überschrifen zwei manuelle Newlines (\\), sonst gibt's komisch Einrückungen
%
\textbf{Geschichte}\\
- Existiert seit Anfang 2013 und gehört der Zerebro Internet GmbH\\
- Hauptsitz in Berlin\\

\subsubsection{Geschäftsmodell}
\textbf{Die Idee}\\
Viele Menschen in Deutschland nutzen keine Kreditkarte und auch kein Onlinebanking. Für diese ist es schwer online 
einzukaufen, da die meisten Anbieter Onlinebanking bzw. eine Kreditkarte voraussetzen. Wegen dieser Tatsache entstand
dieses Projekt. Hierbei können die Kunden online einkaufen und bekommen einen Barcode zurück. Mit Hilfe von diesem
Barcode, kann derjenige dann bei einer Filiale, die Barzahlen unterstützt, das Geld für den Einkauf zahlen. Zeitgleich
wird in diesem Moment dem Online-Shop eine Bestätigung geschickt, dass der Kunde gezahlt hat.\\
\\
\textbf{Vision/Mission/Zukunftsperspektive}\\
Einen bisher wenig genutzten Markt zu bedienen.\\
\\
\textbf{Art und Weise, mit der das Unternehmen Gewinne erwirtschaftet}\\
Genauere Details gibt es hier leider nicht. Auf der Webseite wird erwähnt, dass individuell mit jedem Händler die 
Gebühren festgelegt werden. Für den Kunden entstehen keine Gebühren.\\
\\
\textbf{Nutzenversprechen}\\
Für den Kunden:\\
- es müssen keine persönlichen Daten, Bankkontodaten oder Kreditkartendaten online preisgegeben werden\\
- es wird keine Accounterstellung bzw. Registrierung benötigt\\
- da der Online-Shop in dem Moment, wenn man zahlt, direkt die Informationen übergibt, kann die Ware auch direkt verschickt werden.\\
\\
\textbf{Ertragsmodell}\\
durch anfallende Gebühren, wie oben erwähnt\\
\subsubsection{Strategie}
- Immer mehr Filialen dazu bekommen, um das Angebot großflächig anbieten zu können\\
- mehr Händler als Kunden gewinnen, um die Anzahl der Kunden zu steigern\\
\subsubsection{Kernkompetenzen}
\textbf{Differenzierung}\\
Bietet die Möglichkeit in Online-Shops einzukaufen, ohne eine Kreditkarte oder ein Bankkonto zu nutzen und dies zudem
mit Bargeld zu bezahlen\\
\\
\textbf{Diversifikation}\\
sehr viele Online-Shops möchten ihren Kunden einfache Transaktionen anbieten und viele Einzelhandel-Geschäfte möchten
die PINs verkaufen.\\
\\
\textbf{Kundennutzen}\\
schon beschrieben\\
\\
\textbf{Imitationsschutz}\\
Konzept ist einfach kopierbar. Die Schwierigkeit besteht darin, dass der Eingang der Zahlung sicher übermittelt wird.\\

\subsubsection{Erfolgsfaktoren}
- es werden Kunden gewonnen, die keine der anderen Dienste nutzen wollen\\
- es gibt in diesem Bereich wenig Konkurrenz\\

\subsubsection{Risikofaktoren}
- zu wenig Filialen, an denen man bezahlen kann\\
- zu wenig Shops, die Barzahlen nutzen\\

\subsubsection{Kennzahlen}
- bisher 1425 Verkaufsstellen \(nach der Webseite wurden seit der Gründung mit der Ankündigung, dass alle DM-Drogerien
und mobilcom-debitel Filialen in Berlin Barzahlen unterstützen werden, keine zusätzlichen gefunden\) \\
- Zum Start gab es 50 Online-Shops, die Barzahlen verwenden, jetzt sind es 190\\

\subsubsection{Markt-Konkurrenz-Analyse}
Als Konkurrent mit ähnlichen Vorteilen existiert Paysafecard. Hier sind wohl die Unterschiede, dass man bei der
Paysafecard zuerst in der Filiale die Karte kauft und dann im Shop direkt einkauft und dass es schon mehr verbreitet ist.
Hier wird sich zeigen, welcher Bezahldienst erfolgreicher sein wird.
