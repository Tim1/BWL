\subsection{Barzahlen}
\subsubsection{Das Unternehmen}

\textbf{Eckdaten}\\
\begin{itemize}
\item Neuer Anbieter
\item 1425 Verkaufsstellen
\item nur in Deutschland vertreten
\item 190 Partner im Internet
\end{itemize}
\\
%
% für neue kleine Überschrifen zwei manuelle Newlines (\\), sonst gibt's komisch Einrückungen
%
\textbf{Geschichte}\\
\begin{itemize}
\item Existiert seit Anfang 2013 und gehört der Zerebro Internet GmbH
\item Hauptsitz in Berlin
\end{itemize}
\subsubsection{Geschäftsmodell}
\textbf{Die Idee}\\
Viele Menschen in Deutschland nutzen keine Kreditkarte und auch kein Onlinebanking. Für diese ist es schwer online 
einzukaufen, da die meisten Anbieter Onlinebanking bzw. eine Kreditkarte voraussetzen. Wegen dieser Tatsache entstand
dieses Projekt. Hierbei können die Kunden online einkaufen und bekommen einen Barcode zurück. Mit Hilfe von diesem
Barcode, kann derjenige dann bei einer Filiale, die Barzahlen unterstützt, das Geld für den Einkauf zahlen. Zeitgleich
wird in diesem Moment dem Online-Shop eine Bestätigung geschickt, dass der Kunde gezahlt hat.\\
\\
\textbf{Vision/Mission/Zukunftsperspektive}\\
Einen bisher wenig genutzten Markt zu bedienen.\\
\\
\textbf{Art und Weise, mit der das Unternehmen Gewinne erwirtschaftet}\\
Genauere Details gibt es hier leider nicht. Auf der Webseite wird erwähnt, dass individuell mit jedem Händler die 
Gebühren festgelegt werden. Für den Kunden entstehen keine Gebühren.\\
\\
\textbf{Nutzenversprechen}\\
Für den Kunden:\\
\begin{itemize}
\item es müssen keine persönlichen Daten, Bankkontodaten oder Kreditkartendaten online preisgegeben werden
\item es wird keine Accounterstellung bzw. Registrierung benötigt
\item da der Online-Shop in dem Moment, wenn man zahlt, direkt die Informationen übergibt, kann die Ware auch direkt verschickt werden.
\\
\textbf{Ertragsmodell}\\
durch anfallende Gebühren, wie oben erwähnt\\
\subsubsection{Strategie}
\begin{itemize}
\item Immer mehr Filialen dazu bekommen, um das Angebot großflächig anbieten zu können
\item mehr Händler als Kunden gewinnen, um die Anzahl der Kunden zu steigern
\end{itemize}
\subsubsection{Kernkompetenzen}
\textbf{Differenzierung}\\
Bietet die Möglichkeit in Online-Shops einzukaufen, ohne eine Kreditkarte oder ein Bankkonto zu nutzen und dies zudem
mit Bargeld zu bezahlen\\
\\
\textbf{Diversifikation}\\
sehr viele Online-Shops möchten ihren Kunden einfache Transaktionen anbieten und viele Einzelhandel-Geschäfte möchten
die PINs verkaufen.\\
\\
\textbf{Kundennutzen}\\
schon beschrieben\\
\\
\textbf{Imitationsschutz}\\
Konzept ist einfach kopierbar. Die Schwierigkeit besteht darin, dass der Eingang der Zahlung sicher übermittelt wird.\\

\subsubsection{Erfolgsfaktoren}
\begin{itemize}
\item es werden Kunden gewonnen, die keine der anderen Dienste nutzen wollen
\item es gibt in diesem Bereich wenig Konkurrenz
\end{itemize}
\subsubsection{Risikofaktoren}
\begin{itemize}
\item zu wenig Filialen, an denen man bezahlen kann
\item zu wenig Shops, die Barzahlen nutzen
\end{itemize}
\subsubsection{Kennzahlen}
\begin{itemize}
\item bisher 1425 Verkaufsstellen (nach der Webseite wurden seit der Gründung mit der Ankündigung, dass alle DM-Drogerien
und mobilcom-debitel Filialen in Berlin Barzahlen unterstützen werden, keine zusätzlichen gefunden)
\item Zum Start gab es 50 Online-Shops, die Barzahlen verwenden, jetzt sind es 190
\end{itemize}
\subsubsection{Markt-Konkurrenz-Analyse}
Als Konkurrent mit ähnlichen Vorteilen existiert Paysafecard. Hier sind wohl die Unterschiede, dass man bei der
Paysafecard zuerst in der Filiale die Karte kauft und dann im Shop direkt einkauft und dass es schon mehr verbreitet ist.
Hier wird sich zeigen, welcher Bezahldienst erfolgreicher sein wird.

\subsection{Paypal}
\subsubsection{Das Unternehmen}

\textbf{Eckdaten}\
\begin{itemize}
\item 230 Millionen Mitgliedskonten in 190 Nationen
\item Marktführer des Payment Processing
\item gehört zu eBay
\end{itemize}
\\
%
% für neue kleine Überschrifen zwei manuelle Newlines (\\), sonst gibt's komisch Einrückungen
%
\textbf{Geschichte}\\
\begin{itemize}
\item Existiert seit 1998
\item ursprünglich Firma für Bezahlmethoden und Kryptografie
\item Damit kein betrügerischer Zugriff durch automatische Systeme möglich ist, wurde ein Captcha-System entwickelt, bei welchem Zahlen von einem unscharfen Bild abgeschrieben werden musste (Gausebeck-Levchin-Test)
\item eBay kaufte zuerst Billpoint als Online-Bezahldienst (später eBay Payments), dadurch aber nur für eBay nutzbar geworden, Paypal wurde aber mehr genutzt
\item Paypal war die erste dotcom, die nach dem 11.September einen erfolgreichen Börsengang durchführte
\item Erfolg von Paypal durch Eigenheiten des Zahlungsverkehrs in den USA (dort wurden in der Vergangenheit zum Transferieren von Geld Schecks verwendet, da das bundesstaatenübergreifende Überweisen verboten war, diese konnten aber nicht bei Onlinetransaktionen verwendet werden
\item Oktober 2002 wurde Paypal von eBay für 1,5 Milliarden US-Dollar aufgekauft
\item Paypal war bei mehr als der Hälfte der eBay-Benutzer in Verwendung
\item Ersetzung von eBay Payments durch Paypal
\end{itemize}

\subsubsection{Geschäftsmodell}
\textbf{Die damalige Idee}\\
Ermöglichung von einfachen Überweisungen, auch international, auf einem einfachen Weg. Außerdem mit direktem Zahlungseingang\\
\\
\textbf{Funktionsweise von Paypal für die Kunden}\\
\begin{itemize}
\item Benutzer erstellt sich bei Paypal ein Konto und gibt eine bestimmte Verbindung an, wie das Geld auf das Paypal-Konto kommen soll (Lastschrift, Kreditkarte, Online-Überweisung, vorherige Überweisung auf das Konto)
\item Während einem Bezahlvorgang bei einem Webshop wird man auf Paypal weitergeleitet
\item Dort loggt man sich ein und bestätigt die Zahlung
\item Nun wird das Geld auf das Paypal-Konto des Verkäufers transferiert und man wird zurück zum dem Webshop weitergeleitet
\item Dort bekommt man im Normalfall eine Bestätigung, dass die Transaktion erfolgreich gewesen ist
\end{itemize}
\\
\textbf{Art und Weise, mit der das Unternehmen Gewinne erwirtschaftet}\\
Gebühren:\\ 
\begin{itemize}
\item für private Käufer kostenfrei
\item bei privaten Verkäufern: 1,9\% + 0,35\euro{}
\item bei Händlern: 1,7\% + 0,35\euro{} (5001-25000\euro{}) 1,5\% + 0,35\euro{} (über 25000\euro{})
\item bei Mikrozahlungen: 10\% + 0,10\euro{}
\end{itemize}
\\
\textbf{Nutzenversprechen}\\
Für den Kunden:\\
\begin{itemize}
\item Bezahlen über verschiedene Möglichkeiten (Bankkonto, Kreditkarte, …)
\item Schnelles und einfaches Bezahlen
\item Käuferschutz
\item Händler bekommt keine Kontodaten
\end{itemize}
\\
\textbf{Ertragsmodell}\\
durch anfallende Gebühren, wie oben erwähnt\\
\subsubsection{Strategie}
\begin{itemize}
\item Da Paypal als Online-Bezahldienst schon sehr erfolgreich ist, haben sie sich nun als Ziel gesetzt im Mobile-Payment aktiv zu werden
\item Es soll mit Paypal möglich sein an der Kasse mit dem Smartphone über das Paypal Guthaben elektronisch zu bezahlen
\item In Zukunft will Paypal wohl auch mit Hilfe von diesem Weg immer mehr Kunden bekommen
\end{itemize}
\subsubsection{Kernkompetenzen}
\textbf{Differenzierung}\\
stärkere Verbreitung als Konkurrenz und Marktvorteil, da Paypal zu eBay gehört und dadurch sehr viele Kunden bekommt\\
\\
\textbf{Kundennutzen}\\
schon beschrieben\\
\\
\textbf{Ausschlaggebende Gründe, wieso jemals ein Kunde zur Konkurrenz wechseln könnte}
\begin{itemize}
\item mehr Sicherheit
\item Gebührensystem attraktiver
\item größere Verbreitung des Bezahldiensts
\end{itemize}

\subsubsection{Risikofaktoren}
\begin{itemize}
\item welche Werte/Eigenschaften können das Erreichen der Ziele des Unternehmens verhindern?
\item Konkurrenz wird in Kernkompetenzen besser
\item Konkurrenz bietet mehr Sicherheit
\end{itemize}

\subsubsection{Kennzahlen}
\begin{itemize}
\item im ersten Quartal 2013 gab es 127,7 Mio. aktive Nutzer
\item Gewinn von eBay stieg im ersten Quartal 2013 um 19\% auf 677 Mio. US-Dollar, hauptsächlich wegen Paypal
\end{itemize}
\subsubsection{Markt-Konkurrenz-Analyse}
Für Paypal gibt es hier mehr Konkurrenten, aber Paypal hat den Vorteil, dass sie zu eBay gehören und dadurch sehr viele Kunden durch eBay schon haben.
