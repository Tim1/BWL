\subsection{Barzahlen}
\subsubsection{Das Unternehmen}

\textbf{Eckdaten}
\begin{itemize}
\item Neuer Anbieter
\item 1425 Verkaufsstellen
\item nur in Deutschland vertreten
\item 190 Partner im Internet
\end{itemize}
%
% für neue kleine Überschrifen zwei manuelle Newlines (\\), sonst gibt's komisch Einrückungen
%
\textbf{Geschichte}
Barzahlen existiert seit Anfang 2013 und gehört zur Zerebro Internet GmbH. Ihren Hauptsitz hat die Firma in Berlin.
\subsubsection{Geschäftsmodell}
\textbf{Die Idee}\\
Viele Menschen in Deutschland nutzen keine Kreditkarte und auch kein Onlinebanking. Für diese ist es schwer online 
einzukaufen, da die meisten Anbieter Onlinebanking bzw. eine Kreditkarte voraussetzen. Wegen dieser Tatsache entstand
dieses Projekt. Hierbei können die Kunden online einkaufen und bekommen einen Barcode zurück. Mit Hilfe von diesem
Barcode, kann derjenige dann bei einer Filiale, die Barzahlen unterstützt, das Geld für den Einkauf zahlen. Zeitgleich
wird in diesem Moment dem Online-Shop eine Bestätigung geschickt, dass der Kunde gezahlt hat.\\
\\
\textbf{Vision/Mission/Zukunftsperspektive}\\
Ihre Vision ist es, einen bisher wenig genutzten Markt zu bedienen.\\
\\
\textbf{Art und Weise, mit der das Unternehmen Gewinne erwirtschaftet}\\
Genauere Details gibt es hier leider nicht. Auf der Webseite wird erwähnt, dass individuell mit jedem Händler die 
Gebühren festgelegt werden. Für den Kunden entstehen keine Gebühren.\\
\\
\textbf{Nutzenversprechen}\\
Für den Kunden:\\
\\
Der Kunde muss bei Barzahlen keine persönlichen Daten, Bankkontodaten oder Kreditkartendaten online preisgeben. Außerdem wird keine Accounterstellung bzw. Registrierung benötigt. Zusätzlich versprechen sie, dass der Online-Shop in dem Moment, wenn man bei einer Filiale bezahlt, direkt darüber benachrichtigt wird.\\
\textbf{Ertragsmodell}\\
Wie oben erwähnt, finanzieren sie sich über anfallende Gebühren.
\subsubsection{Strategie}
Ihre Strategie ist es, immer mehr Filialen dazu zu bekommen, um ihr Angebot großflächig anbieten zu können. Zusätzlich müssen sie auch noch mehr Händler als Kunden gewinnen.
\subsubsection{Kernkompetenzen}
\textbf{Differenzierung}\\
Barzahlen bietet die Möglichkeit in Online-Shops einzukaufen, ohne eine Kreditkarte oder ein Bankkonto zu nutzen und dies zudem
mit Bargeld zu bezahlen.\\
\\
\textbf{Diversifikation}\\
Sehr viele Online-Shops möchten ihren Kunden einfache Transaktionen anbieten und viele Einzelhandel-Geschäfte möchten
die PINs verkaufen.\\
\\
\textbf{Kundennutzen}\\
schon beschrieben\\
\\
\textbf{Imitationsschutz}\\
Das Konzept ist einfach kopierbar. Die Schwierigkeit besteht darin, dass der Eingang der Zahlung sicher übermittelt wird.

\subsubsection{Erfolgsfaktoren}
Es werden Kunden gewonnen, die keine der anderen Dienste nutzen wollen. Außerdem gibt es in diesem Bereich wenig Konkurrenz
\subsubsection{Risikofaktoren}
Da es momentan zu wenig Filialen gibt, die Barzahlen unterstützen, könnte das noch ein Problem werden. Außerdem gibt es auch bisher wenig Shops, die Barzahlen nutzen.
\subsubsection{Kennzahlen}
\begin{itemize}
\item bisher 1425 Verkaufsstellen (nach der Webseite wurden seit der Gründung mit der Ankündigung, dass alle DM-Drogerien
und mobilcom-debitel Filialen in Berlin Barzahlen unterstützen werden, keine zusätzlichen gefunden)
\item Zum Start gab es 50 Online-Shops, die Barzahlen verwenden, jetzt sind es 190.
\end{itemize}

\subsection{Paypal}
\subsubsection{Das Unternehmen}

\textbf{Eckdaten}
\begin{itemize}
\item 230 Millionen Mitgliedskonten in 190 Nationen
\item Marktführer des Payment Processing
\item gehört zu eBay
\end{itemize}
%
% für neue kleine Überschrifen zwei manuelle Newlines (\\), sonst gibt's komisch Einrückungen
%
\textbf{Geschichte}
Paypal existiert seit 1998 und war ursprünglich eine Firma für Bezahlmethoden und Kryptografie. Damit kein betrügerischer Zugriff durch automatische Systeme möglich ist, wurde ein Captcha-System entwickelt, bei welchem Zahlen von einem unscharfen Bild abgeschrieben werden musste (Gausebeck-Levchin-Test). eBay kaufte zuerst Billpoint als Online-Bezahldienst (später eBay Payments), welche aber dadurch nur für eBay nutzbar worden. Paypal wurde aber mehr genutzt. Paypal war die erste dotcom, die nach dem 11.September einen erfolgreichen Börsengang durchführte. Da durch Eigenheiten des Zahlungsverkehrs in den USA das bundesstaatenübergreifende Überweisen verboten war und hierfür in der Vergangenheit daher zum Transferieren von Geld Schecks verwendet wurden, hat Paypal hier eine gute Alternative geboten. Außerdem kann man Schecks nicht für Onlinetransaktionen nutzen. Dies hat ihnen sehr zum Erfolg geholfen. Oktober 2002 wurde Paypal dann von eBay für 1,5 Milliarden US-Dollar aufgekauft. Davor waren sie schon bei mehr als der Hälfte der eBay-Benutzer in Verwendung. Später wurde dann eBay Payments durch Paypal ersetzt.
\subsubsection{Geschäftsmodell}
\textbf{Die damalige Idee}\\
Damals war es die Idee einfachen Überweisungen, auch international, auf einem einfachen Weg zu ermöglichen. Dies sollte außerdem über einen direkten Zahlungseingang funktionieren.\\
\\
\textbf{Funktionsweise von Paypal für die Kunden}
Der Benutzer erstellt sich bei Paypal ein Konto und gibt eine bestimmte Verbindung an, wie das Geld auf das Paypal-Konto kommen soll (Lastschrift, Kreditkarte, Online-Überweisung, vorherige Überweisung auf das Konto). Während einem Be-zahlvorgang wird man dann bei einem Webshop auf Paypal weitergeleitet. Dort loggt man sich ein und bestätigt die Zahlung. Nun wird das Geld auf das Paypal-Konto des Verkäufers transferiert und man wird zurück zum dem Webshop weitergeleitet. Dort bekommt man dann im Normalfall eine Bestätigung, dass die Transaktion erfolgreich gewesen ist.\\
\textbf{Art und Weise, mit der das Unternehmen Gewinne erwirtschaftet}\\
Gebühren:
\begin{itemize}
\item für private Käufer kostenfrei
\item bei privaten Verkäufern: 1,9\% + 0,35\euro{}
\item bei Händlern: 1,7\% + 0,35\euro{} (5001-25000\euro{}) 1,5\% + 0,35\euro{} (über 25000\euro{})
\item bei Mikrozahlungen: 10\% + 0,10\euro{}
\end{itemize}
\textbf{Nutzenversprechen}\\
Für den Kunden:\\
Paypal verspricht dem Kunden schnelles und einfaches Bezahlen über verschiedene Möglichkeiten wie Bankkonto, Kreditkarte, etc. Dabei existiert ein Käuferschutz und der Händler bekommt keine Kontodaten des Kunden.\\
\textbf{Ertragsmodell}\\
Wie oben erwähnt, finanzieren sie sich über anfallende Gebühren.
\subsubsection{Strategie}
Da Paypal als Online-Bezahldienst schon sehr erfolgreich ist, haben sie sich nun als Ziel gesetzt im Mobile-Payment aktiv zu werden. Es soll mit Paypal möglich sein, an der Kasse mit dem Smartphone über das Paypal Guthaben elektronisch zu bezahlen. In Zukunft will Paypal wohl auch mit Hilfe von diesem Weg immer mehr Kunden bekommen.
\subsubsection{Kernkompetenzen}
\textbf{Differenzierung}\\
Paypal hat eine stärkere Verbreitung als die Konkurrenz und einen Marktvorteil, da Paypal zu eBay gehört und dadurch sehr viele Kunden bekommt.\\
\\
\textbf{Kundennutzen}\\
schon beschrieben\\
\\
\textbf{Ausschlaggebende Gründe, wieso jemals ein Kunde zur Konkurrenz wechseln könnte}
\begin{itemize}
\item mehr Sicherheit
\item Gebührensystem attraktiver
\item größere Verbreitung des Bezahldiensts
\end{itemize}

\subsubsection{Risikofaktoren}
Die Konkurrenz wird in den Kernkompetenzen besser wie Paypal und bietet auch mehr Sicherheit, was zu einem Verlust an Kunden führen könnte

\subsubsection{Kennzahlen}
\begin{itemize}
\item im ersten Quartal 2013 gab es 127,7 Mio. aktive Nutzer
\item Gewinn von eBay stieg im ersten Quartal 2013 um 19\% auf 677 Mio. US-Dollar, hauptsächlich wegen Paypal
\end{itemize}
