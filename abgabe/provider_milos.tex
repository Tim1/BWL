\subsection{Paysafecard}
\subsubsection{Das Unternehmen}


\textbf{Eckdaten}

\begin{itemize}
        \item europäischer Marktführer bei online Prepaid Zahlungsmitteln
        \item 450.000 Verkaufsstellen
        \item in 33 Ländern vertreten
        \item 4000 Partner-Webshops im Internet aus Bereichen Games, Social Media \& Communities, Musik, Film \& Entertainmeint
        \item das erste Produkt der paysafecard group
\end{itemize}


\textbf{Geschichte}

\begin{itemize}
        \item existiert seit dem Jahr 2000 und gehört der paysafecard group.
        \item 2001 übernahm Paysafecard den Anbieter Wallie.
        \item erste bankenrechtlich genehmigte Online-Zahlungsmittel in Europa.
        \item Karte von PrePaid Services Company Limited herausgegeben und verwaltet.
        \item Hauptsitz in Wien, London und New York.
        \item wird seit 2008 von der britischen Financial Services Authority reguliert.
        \item Februar 2013 wurde paysafecard group von dem britischen Online-Bezahldienst Skrill, ehemals Moneybookers, übernommen.
\end{itemize}

\subsubsection{Geschaeftsmodell}
\textbf{Die Idee}

\begin{itemize}
        \item Im Internet gibt es keine Zahlungsmöglichkeit, die wie Bargeld funktioniert, also ohne Bankenkonto und ohne Kreditkarte. Diese Lücke wurde von paysafecard geschlossen.?
        \item Vision/Mission/Zukunftsperspektive:
weltweit größte Anbieter von Prepaid Zahlungsmittel im Internet werden. Moneybookers, übernommen.
\end{itemize}

\textbf{Funktionsweise von paysafecard für die Endkunden}

\begin{itemize}
        \item Benutzer erwirbt an einer Verkaufsstelle (Tankstelle, Kiosk, Post, Lotto-Annahmestelle, Lebensmitteleinzelhandel) eine paysafecard mit einem Guthaben im Wert von 10, 15, 20, 25, 30, 50 oder 10 Euro, das ihm in Form einer 16-stelligen PIN ausgehändigt wird.
        \item PIN wird während des Bezahlvorgangs bei einem Webshop angegeben.
        \item ist das Guthaben der paysafecard aufgebraucht, wird die jeweilige PIN ungültig. Eine neue Karte mit neuen PIN muss erworben werden.
        \item reicht für die Transaktion eine Karte nicht aus, kann eine zweite hinzugezogen werden.
        \item Auszahlung des noch auf der Karte befindlichen Guthabens jederzeit möglich.
        \item als registrierter Kunde kann man bis zu zwei 1000 Euro Karten besitzen, aber maximales Guthaben darf nicht 2500 Euro überschreiten.
        \item kostenlos ist:
        \item   Kartenausgabe, auch die zweite Ausgabe bei Defekt oder Diebstahl
        \item   Guthabenabfrage über das Internet
        \item   Transaktionsübersicht über das Internet
        \item   bei Bezahlung in Euro falle keine Gebühren an
\end{itemize}


\textbf{Art und Weise, mit der das Unternehmen Gewinne erwirtschaftet}

\begin{itemize}
        \item auf Währungen ungleich den Euro werden ein Umrechnungsaufschlag von 2\% des Transaktionsvolumens erhoben.
        \item nach Ablauf der ersten 12 Monate ab Kauf berechnet das Unternehmen eine Bereitstellungsgebühr von 2 Euro pro Monat.
        \item Auszahlung des noch auf der Karte befindlichen Guthabens ist jederzeit möglich, aber es fällt eine Gebühr von 7,50 Euro an, wenn der Rücktausch vor Ablauf oder nach mehr als einem Jahr nach Ablauf des Vertrages verlangt wird.
        \item auf der Händlerseite fällt ein transatkionsabhängiges Disagio an:
        \item   berechnet aus Höhe des monatlichen Umsatzes mit Paysafecard und nach Branche. In der Regel : 8-19,5\% Transaktionsgebühren für den Händler.
\end{itemize}


\textbf{Nutzenversprechen für die Endkunden}
\begin{itemize}
        \item zahlt ohne persönliche Daten, Bankkontodaten oder Kreditkarte, d.h der Kund bleibt anonym.
        \item bietet somit Schutz vor Identitätsdiebstahl bzw. Phishing
        \item schnell und einfacher Zahlungsvorgang
        \item bietet Ausgabenkontrolle: paysafecard setzt dem Kunden ein Limit und fördert das Bewusstsein, wie viel Geld zur Verfügung steht.
\end{itemize}


\textbf{Nutzenversprechen für die Händler als Verkaufsstelle bzw. Vertriebspartner}
\begin{itemize}
        \item steigern ihren Umsatz, da sie an jeder verkauften paysafecard eine Provision erhalten
        \item erhöhen ihre Bekanntheit, da sie auf der paysafecard-Produktseite beworben wird
        \item steigern die Kundenfrequenz, da paysafecard genutzt wird. Gewinn von neuen Stammkunden
\end{itemize}

\textbf{Nutzenversprechen für die Webshops}
\begin{itemize}
        \item keine Chargebacks: Rückbelastung von Kartentransaktionen, die entstehen, weil ein Karteninhaber oder seine Bank mit einer Transaktion nicht einverstanden ist. Keine Rückabwicklungskosten
        \item Zielgruppe erweitern:
        \item   Kunden, die anonym bleiben wollen, können dann einkaufen
        \item   die von ihrer Bank keine Kreditkarte erhaltenen
        \item   Junge Menschen, die noch keinen Zugang zu den klassischen Bezahlsystemen wie Kreditkarte haben
        \item   Kunden, die gerne cash im Internet bezahlen möchten.
einfaches Handling: Implementierung der Bezahlmöglichkeit ist sehr einfach.
\end{itemize}

\textbf{Architektur der Wertschöpfung}
\begin{itemize}
        \item Die Karten haben ein Guthaben von 10,15, 20, 25, 30, 50 und 100 Euro.
        \item Betragsgrenze für nicht registrierte Kunden: 100 Euro
        \item Betragsgrenze für registrierte Kunden: 1000 Euro
        \item Provision an Partner-Verkaufsstellen
\end{itemize}


\textbf{Ertragsmodell}
\begin{itemize}
        \item durch die anfallenden Gebühren, die zuvor im Kapitel Geschäftsmodell beschrieben wurden

\end{itemize}

\subsubsection{Strategie}
\begin{itemize}
        \item meist langfristig geplante Verhaltensweisen der Unternehmen zur Erreichung ihrer Ziele
        \item Strategie identifizieren und etwas darüber schreiben..

        \item es gibt eine Vision, auf die sich paysafecard bezieht und darauf hin arbeitet
        Um die Vision zu verwirklichen, muss sich paysafecard ausbreiten, d.h.  noch mehr Kunden für sich gewinnen und deren Transaktionsvolumen        erhöhen (mehr Kunden, die noch mehr Geld als zuvor transferieren):
        \item weitere Verkaufsstellen anbieten, sodass es die Karten überall erhältlich sind
        \item weitere Partnerschaften mit Online-Shops bilden um somit paysafecard überall präsent zu haben. man kann überall mit paysafecard zahlen?


\end{itemize}

\subsubsection{Kernkompetenzen}

\textbf{Differenzierung}
\begin{itemize}
        \item bietet für die Kunden ein sicheres, anonymes und einfaches Prepaid-Online-Bezahlverfahren an, das in vielen Ländern von vielen Online-Shops akzeptiert wird.
\end{itemize}

\textbf{Diversifikation}
\begin{itemize}
        \item sehr viele Online-Shops möchten ihren Kunden einfache Transaktionen anbieten und viele Einzelhandel-Geschäfte möchte die PINs verkaufen, und zwar weltweit. Die schwierigste Hürde liegt in den unterschiedlichen Gesetzeslagen der jeweiligen Länder. Ist diese überwunden, kann paysafecard im Land schnell verbreitet werden.
\end{itemize}

\textbf{Imitationsschutz}
\begin{itemize}
        \item ist einfach kopierbar, die Schwierigkeit liegt an ein IT-sicheres Transaktionsverfahren zu entwickeln.
        \item Ausschlaggebend Gründe, wieso jemals ein Kunde zur Konkurrenz wechseln würde:
        \item   Verfügbarkeit in Onlineshops und Einzelhandel besser
        \item   Sicherheit der Guthaben besser
        \item   Anonymität gewährleistest
        \item   Gebührensystem attraktiver
\end{itemize}

\textbf{Kundennutzen}
\begin{itemize}
        \item schon beschrieben

\end{itemize}



\subsubsection{Erfolgsfaktoren}
\begin{itemize}
        \item Die Strategieführung des Unternehmens ist ein Erfolgsfaktor. Sobald das Produkt von den Kunden genau ihre Anforderungen erfüllt und übertrefft, wird es weitergenutzt und weitere Neukunden anziehen.
\end{itemize}


\subsubsection{Risikofaktoren}
\begin{itemize}
        \item Konkurrenz wird in Kernkompetenzen besser
        \item Politik:
        
        \begin{itemize}
       		\item   entscheidet in den jeweiligen Land die Gesetze und auch über das Prepaid-Online-Bezahlverfahren paysafecard.
        	\item   Bei Sicherheitslücken oder Verstößen von Gesetzen kann paysafcard im jeweiligen Land benachteiligt werden, was zu Kundenverlust führen kann
        	\item   generelle Probleme (technische, politische, rechtliche), die in eines der Länder entstehen können, kann die Kunden aus den anderen Ländern evtl. zum Nachdenken bringen und zur Konkurrenz wechseln lassen.
        \end{itemize}
\end{itemize}



\subsubsection{Kennzahlen}

\textbf{Erfolgskennzahlen}
\begin{itemize}
        \item Umsatz: 16,3 Mio Euro
        \item 2 Mio. Kunden bezahlen monatlich mit paysafecard (Video)
        \item Anzahl der Transaktionen mit paysafecard:
        \item   2000: 313    ..  135.992   .. 2,3Mio.  20,2Mio... Euro
        \item   2012: 55 Mio.
        \item Anzahl der paysafecard Verkaufsstellen
        \item   2000: 1000.....5000...50150....249720...
        \item   2012: 450000
        \item weltweit verkaufte PINs
        \item   2000: 470...108242...2,3 Mio.....19 Mio....
        \item   2012: 50 Mio.
\end{itemize}


\subsubsection{Kennzahlen}

\begin{itemize}
        \item Ukash
        \item Bitcoin
\end{itemize}