\subsection[Paysafecard]{Paysafecard\footnote{\url{http://de.wikipedia.org/wiki/Paysafecard}}}
\subsubsection{Das Unternehmen}
Das im Jahre 2000 gegründete Unternehmen bietet den Kunden das gleichnamige Produkt Paysfafecard an. Es ist ein elektronisches Zahlungsmittel, das nach dem Prepaid-System funktioniert. Der Kunde kauft die Karte bei einer der 450.000 Verkaufsstellen um mit dieser auf Webshops anonym, sicher und bequem bezahlen. Die Karte hat genau denselben realen Wert, mit der sie gekauft wurde. Nach dem neuesten Stand akzeptieren Paysafecard über 4.000 Webshops aus den Bereichen Games, Social Media \& Communities, Musik, Film \& Entertainment an. Vertreten ist Paysafecard weltweit in über 30 Ländern.



\subsubsection{Geschäftsmodell}

\paragraph{Idee}\label{idee}

Vorstandsvorsitzender Michael Müller erklärt im Video die Geschäfts\-idee und die Vision von Paysafecard:
\begin{quote}
``Hinter der Gründung von Paysafecard steckt die Erfahrung, dass im Internet keine Zahlungsmöglichkeit gab, die wie Bargeld funktioniert, also ohne Bankenkonto und ohne Kreditkarte. Diese Lücke haben wir geschlossen.''
\end{quote}
Die Vision des Unternehmens ist es, der weltweit größter Anbieter von Pre\-paid-Zahlungsmittel im Internet zu werden.\footnote{\url{ http://www.paysafecardgroup.com/video.html}}


\paragraph{Funktionsweise von Paysafecard für die Endkunden}

Der Benutzer erwirbt an einer Verkaufsstelle (Tankstelle, Kiosk, Post, Lotto-Annahmestelle, Lebensmitteleinzelhandel) eine Paysafecard mit einem Guthaben im Wert von \EUR{10}, \EUR{15}, \EUR{20}, \EUR{25}, \EUR{30}, \EUR{50} oder \EUR{100}, das ihm in Form einer 16-stelligen PIN ausgehändigt wird. Die PIN wird während des Bezahlvorgangs bei einem Webshop angegeben. Ist das Guthaben der Paysafecard aufgebraucht, wird die jeweilige PIN ungültig. Eine neue Karte mit neuen PIN muss erworben werden. Reicht für die Transaktion eine Karte nicht aus, kann eine Zweite hinzugezogen werden. Auszahlung des noch auf der Karte befindlichen Guthabens ist jederzeit möglich. 
Als registrierter Kunde kann man bis zu zwei \EUR{1000}-Karten besitzen, wobei das maximale Guthaben nicht \EUR{2500} überschreiten darf.
\\
Kostenlos sind folgende Dienste: 
\begin{itemize}
	\item   Kartenausgabe, auch die zweite Ausgabe bei Defekt oder Diebstahl
    \item   Guthabenabfrage über das Internet
    \item   Transaktionsübersicht über das Internet
    \item   bei Bezahlung in Euro, fals keine Gebühren an
\end{itemize}


\paragraph{Ertragsmodell}

Paysafecard erwirtschaftet seine Gewinne durch Gebühren, die bei der Benutzung des Bezahldienstes entstehen. Auf Währungen un\-gleich dem Euro wird ein Umrechnungsaufschlag von 2\% des Transaktions\-vo\-lumens erhoben. Nach Ablauf der ersten 12 Monate ab Kauf berechnet das Unternehmen eine Bereitstellungsgebühr von \EUR{2} pro Monat. Die Auszahlung des noch auf der Karte befindlichen Guthabens ist jederzeit möglich, aber es fällt eine Gebühr von \EUR{7,50} an, wenn der Rücktausch vor Ablauf oder nach mehr als einem Jahr nach Ablauf des Vertrages verlangt wird. Auf der Händlerseite fällt ein transatkionsabhängiger Abschlag an, welches sich aus Höhe des monatlichen Umsatzes mit Paysafecard und nach Branche berechnet. In der Regel sind es 8-19,5\% Transaktionsgebühren für den Händler.



\subsubsection{Strategie}

Paysafecard hat eine Vision, auf die sie sich bezieht und darauf hin arbeitet. Diese wurde im Kapitel \ref{idee} beschrieben. Um die Vision zu verwirklichen, muss Paysafecard noch mehr Kunden für sich gewinnen und deren Transaktionsvolumen erhöhen. Das Unternehmen wird versuchen, weitere Verkaufsstellen als Vertriebspartner zu finden, sodass die Karten für die Kunden unmittelbar in der Nähe erhältlich sind. Außerdem müssen weitere Partnerschaften mit Online-Shops gebildet werden um somit Paysafecard als universelles Zahlungsmittel zu vermarkten. Ziel ist es mit einer Karte, der Paysafecard, alle Online-Transaktionen durchzuführen.


\subsubsection{Kernkompetenzen}

\paragraph{Differenzierung}
Paysafecard bietet seinen Kunden ein sicheres, anonymes und einfaches Prepaid-Online-Bezahlverfahren an, das in vielen Ländern von vielen Online-Shops akzeptiert wird.


\paragraph{Diversifikation}
Sehr viele Online-Shops möchten ihren Kunden einfache Transaktionen anbieten und viele Einzelhandel-Geschäfte möchte die PINs verkaufen, und zwar weltweit. Paysafecard ist somit für jeden Verkaufsmarkt interessant. Die schwierigste Hürde liegt in den unterschiedlichen Gesetzeslagen der jeweiligen Länder. Ist diese überwunden, kann Paysafecard im Land schnell vermarktet werden.

\paragraph{Imitationsschutz}
Das Konzept des Prepaid-Bezahlsystems ist einfach kopierbar. Die Schwierigkeit liegt darin, ein IT-sicheres Transaktionsverfahren zu entwickeln und es weltweit zur Verfügung stehen zu können.

\paragraph{Kundennutzen}
\subparagraph[Nutzerversprechen]{Nutzerversprechen für die Endkunden\footnote{\url{ http://www.paysafecardgroup.com/verantwortung/das-prepaid-prinzip.html}}}
Der Kunde bezahlt ohne persönliche Daten, Bankkontodaten oder Kreditkarte und bleibt somit ano\-nym. Identitätsdiebstahl bzw. Phishing kann nicht stattfinden, da nur ein PIN angegeben werden muss. Der Zahlungsvorgang ist für den Kunden bequem und einfach ausführbar und bietet Ausgabenkontrolle: Es setzt ihm ein Limit durch das Prepaid-System und fördert das Bewusstsein, wie viel Geld zur Verfügung steht.


\subparagraph[Nutzerversprechen für die Händler als Verkaufsstelle bzw. Vertriebspartner]{Nutzerversprechen für die Händler als Verkaufsstelle bzw. Vertriebspartner \footnote{\url{ https://www.paysafecard.com/de-global/business/verkaufsstellen/vorteile/}}}
Paysafecard wirbt damit, dass Vertriebspartner, die die Karten verkaufen, ihren Umsatz steigern, weil sie an jeder verkauften Paysafecard eine Provision erhalten. Die Händler erhöhen ihre Bekanntheit weltweit, denn sie werden auf der Paysafecard-Produktseite als Vertriebspartner beworben. Außerdem steigert Paysafecard die Kundenfrequenz, weil Kunden diese Karte kaufen wollen. Es ermöglicht den Händlern neue Stammkunden zu gewinnen.


\subparagraph[Nutzerversprechen für die Webshops]{Nutzerversprechen für die Webshops \footnote{\url{ https://www.paysafecard.com/de-global/business/webshops/paysafecard-akzeptieren/}}}
Für die Webshops entstehen keine Rückbelastung von Kartentransaktionen, falls ein Karteninhaber oder seine Bank mit einer Transaktion nicht einverstanden ist. Die Zielgruppe kann erweitert werden, denn es können Kunden einkaufen, die: 
\begin{itemize}
	\item anonym bleiben wollen,
    \item von ihrer Bank keine Kreditkarte erhaltenen,
    \item noch keinen Zugang zu den klassischen Bezahlsystemen wie Kreditkarte haben oder
    \item gerne cash im Internet bezahlen möchten.
\end{itemize}



\subsubsection{Erfolgsfaktoren}
Die Strategieführung des Unternehmens ist ein Erfolgsfaktor. Sobald das Produkt von den Kunden genau ihre Anforderungen erfüllt und übertrefft, wird es weitergenutzt und weitere Neukunden anziehen.



\subsubsection{Risikofaktoren}
Das generelle Risiko ist, wenn die Konkurrenz in Kernkompetenzen besser wird als Paysafecard. Ausschlaggebende Gründe wieso ein Paysafecard-Kunde zur Konkurrenz wechseln könnte sind:
\begin{itemize}
	\item hohe Verfügbarkeit in Onlineshops und Einzelhandel,
	\item IT-Sicherheitskonzept ist sicherer, der Kunde vertraut der Konkurrenz mehr,
	\item attraktiveres Gebühren- bzw. Provisionssystem
\end{itemize}

Ein weiterer bedeutender Risikofaktor stellt die Politik in den Ländern dar, in denen Paysafecard vertreten ist bzw. in Zukunft vertreten sein will. Durch das Anonymisieren der Kunden können kriminelle Handlungen nicht verfolgt werden, beispielsweise Geldwäscherei. Somit kann es dazu kommen, dass durch neue Gesetze Paysafecard in den jeweiligen Ländern benachteiligt, was zu Kundenverlust und Gewinneinbrüche führen kann.
\\
In Deutschland sind neue Gesetzte zur Optimierung der Geldwäschepräven\-tion am 28.12.2011 in Kraft getreten. Für Paysafecard in Deutschland bedeutet es, dass ab dem 5.09.2012 Aufladungen von paysafecard Bereichen und Neuregistrierungen nicht mehr möglich sind. Ab dem 15.04.2013 können keine Bezahlungen mit my paysfaceard in Deutschland getätigt werden. Dies führte zu starken Verlusten von Kunden.\footnote{\url{http://blog.paysafecardgroup.com/de/e-geld-in-deutschland-veranderungen-fur-paysafecard/}}
\\
Es stellt sich die Frage, ob weitere Länder Gesetzte hinsichtlich der Online-Kriminalität verschärfen wollen.
\\
Ein weiteres Risiko besteht, wenn die technische Sicherheitslücken im Bezahl\-verfahren entdeckt werden, die nicht sofort geschlossen werden können: Das Unternehmen verliert das Vertrauen der Kunden und den Vertragspartnern. Der Gedanke, zur Konkurrenz zur wechseln, erhöht sich immens.
 

\subsubsection{Kennzahlen}

\paragraph{Erfolgskennzahlen}
\begin{itemize}

	\item Umsatz 2011: \EUR{16,3 }Mio \footnote{\url{http://wirtschaftsblatt.at/archiv/1265771/Skrill-kauft-Paysafecard-fuer-140-Millionen}}
	\item Kennzahlen aus dem Marketingvideo von Paysafecard\footnote{\url{http://www.paysafecardgroup.com/video.html}}
	\begin{itemize}
	
	  	\item 2 Mio. Kunden bezahlen monatlich mit Paysafecard
    	\item Anzahl der Transaktionen mit Paysafecard:
    	\begin{itemize}
    	  	\item   2000: 313
    		\item   2012: 55 Mio.
    	\end{itemize}
	  
	    \item Anzahl der Vertriebshändler:
	    \begin{itemize}
	     	\item   2000: 1000
	    	\item   2012: 450.000
	    \end{itemize}
	   
	    \item weltweit verkaufte PINs:
	    \begin{itemize}
	     	\item   2000: 470 
	    	\item   2012: 50 Mio.
	    \end{itemize}
	    
	\end{itemize}
	
\end{itemize}
